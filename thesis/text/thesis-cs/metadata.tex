%%% Vyplňte prosím základní údaje o závěrečné práci.
%%% Automaticky se pak vloží na všechna místa, kde jsou potřeba.

% Druh práce:
%	"bc" pro bakalářskou
%	"mgr" pro diplomovou
%	"phd" pro disertační
%	"rig" pro rigorozní
\def\ThesisType{bc}

% Název práce v jazyce práce (přesně podle zadání)
\def\ThesisTitle{\xxx{Optimalizace plánu směn pohotovostních služeb}}

% Název práce v angličtině
\def\ThesisTitleEN{\xxx{Emergency services shift plan optimization}}

% Jméno autora (vy)
\def\ThesisAuthor{\xxx{Tomáš Arnold Tillmann}}

% Rok odevzdání
\def\YearSubmitted{\xxx{2024}}

% Název katedry nebo ústavu, kde byla práce oficiálně zadána
% (dle Organizační struktury MFF UK:
% https://www.mff.cuni.cz/cs/fakulta/organizacni-struktura,
% případně plný název pracoviště mimo MFF)
\def\Department{\xxx{Katedra distribuovaných a spolehlivých systémů}}
\def\DepartmentEN{\xxx{Department of Distributed and Dependable Systems}}

% Jedná se o katedru (department) nebo o ústav (institute)?
\def\DeptType{\xxx{Department}}
\def\DeptTypeEN{\xxx{Department}}

% Vedoucí práce: Jméno a příjmení s~tituly
\def\Supervisor{\xxx{Adam Šmelko}}

% Pracoviště vedoucího (opět dle Organizační struktury MFF)
\def\SupervisorsDepartment{\xxx{Katedra distribuovaných a spolehlivých systémů}}
\def\SupervisorsDepartmentEN{\xxx{Department of Distributed and Dependable Systems}}

% Studijní program (kromě rigorozních prací)
\def\StudyProgramme{\xxx{Informatika}}

% Nepovinné poděkování (vedoucímu práce, konzultantovi, tomu, kdo
% vám nosil pizzu a vařil čaj apod.)
\def\Dedication{%
\xxx{
Rád bych vyjádřil své upřímné poděkování všem, kteří mi pomohli a podporovali mě při psaní této práce.
V první řadě děkuji svému vedoucímu práce, Mgr. Adamu Šmelkovi, za cenné rady, trpělivost a odborné vedení, které mi během celého procesu tvorby této práce poskytl.
Jeho znalosti a zkušenosti byly pro mě neocenitelné.
Dále bych rád poděkoval své rodině za jejich podporu.
Velké poděkování patří také mým přátelům za jejich pomoc, podporu a inspiraci.
}
}

% Abstrakt (doporučený rozsah cca 80-200 slov; nejedná se o zadání práce)
\def\Abstract{%
\xxx{
  Nalezení optimálního naplánování týmů záchranářů a záchranných vozidel je ústředním problémem každé pohotovostní služby.
  Pohotovostní služba by při optimálním naplánování měla být schopna úspěšně odbavit co největší počet incidentů za co nejmenší vynaložené náklady v rámci daného časového úseku, například jednoho dne.
  Tato práce se zabývá metodami, které úspěšně naleznou taková optimální naplánování.
  Nalezené metody jsou aplikovány a zkoumány na pražské pohotovostní službě.
  Následně jsou mezi sebou porovnány a na závěr je rozhodnuto, jaké metody umí nalézt nejlepší naplánování, a jsou tak nejvhodnějšími pro praktické využití.
  }
}

% Anglická verze abstraktu
\def\AbstractEN{%
\xxx{Abstract.}
}

% 3 až 5 klíčových slov (doporučeno) oddělených \sep
% Hodí se pro nalezení práce podle tématu.
\def\ThesisKeywords{%
\xxx{Optimalizace \sep Pohotovostní služba \sep Záchranná služba \sep Optimální plán}
}

\def\ThesisKeywordsEN{
\xxx{Optimization \sep Emergency service \sep Optimal plan}
}

% Pokud některá z položek metadat obsahuje TeXové řídící sekvence, je potřeba
% dodat i verzi v obyčejném textu, která se objeví v metadatech formátu XMP
% zabudovaných do výstupního souboru PDF. Pokud si nejste jistí, prohlédněte si
% vygenerovaný soubor thesis.xmpdata.
\def\ThesisAuthorXMP{\ThesisAuthor}
\def\ThesisTitleXMP{\ThesisTitle}
\def\ThesisKeywordsXMP{\ThesisKeywords}
\def\AbstractXMP{\Abstract}

% Máte-li dlouhý abstrakt a nechceme se mu vejít na informační stranu,
% můžete použít toto nastavení ke zmenšení písma informační strany.
% (Uvažte nicméně zkrácení abstraktu, to je často lepší.)
\def\InfoPageFont{}
%\def\InfoPageFont{\small}  % odkomentujte pro zmenšení písma
