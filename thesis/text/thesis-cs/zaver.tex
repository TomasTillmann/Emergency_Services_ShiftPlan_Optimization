\chapter*{Závěr}
\addcontentsline{toc}{chapter}{Závěr}

Tato práce se zabývá problémem nalezení optimálního plánu pohotovostní služby. 
V první kapitole jsme problém zformalizovali a důkladně zanalyzovali, abychom zvolili vhodné metody řešení.
Ukázalo se, že nejvhodnější je problém modelovat jako optimalizační úlohu s jednou účelovou funkcí.
Účelová funkce je vhodným složením ceny plánu a počtu úspěšně odbavených incidentů.
Abychom věrohodně zjistili počet úspěšně odbavených incidentů, spustili jsme diskrétní simulaci, která pro danou sadu incidentů napodobuje chování plánu v průběhu jednoho dne.
Použití simulace omezilo možné techniky, které se běžně používají pro řešení těžkých optimalizačních problému, jako například lineární programování.
Na druhou stranu, umožňuje věrohodně získat počet úspěšně odbavených incidentů, stejně jako by plán již skutečně odbavoval incidenty v terénu.
 
V druhé kapitole jsme diskutovali několik různých metod, od využití dynamického programování, až po použití metaheuristických přístupů.
Nalezli jsme rekurzivní vztah mezi plány odbavující o jedna méně incidentů
a využili jej pro řešení úlohy pomocí dynamického programování. Přestože výpočetní složitost je exponenciální vůči počtu incidentů, 
jedná se o dramatické zlepšení oproti naivnímu řešení, které jenom náhodně vytváří plány.

Ve třetí kapitole jsme nalezené metody aplikovali na konkrétní pohotovostní službu, a to sice na pohotovostní službu hlavního města Prahy.
Synteticky jsme vygenerovali sadu incidentů, která vhodně reprezentuje možnou skutečnou situaci.
Na základě volně dostupných dat pražské pohotovostní služby představuje, jak by se mohly incidenty v průběhu dne skutečně odehrát.

Při aplikaci metod jsme zjistili, že některé metody jsou vhodnější než jiné. Například se ukázalo, že nemá smysl používat tabu prohledávání, jelikož je obtížné nalézt i jedno lokální optimum.
Na druhou stranu, kombinace lokálního prohledávání nebo simulovaného žíhání spolu s dynamickým programováním se ukázaly jako nejlepší metody pro praktické využití.
Nalezené plány uměly úspěšně odbavit všechny incidenty jen s polovinou vozidel a do stovky týmů záchranářů.
Nalezené plány jsou kvalitní, a jistě by se uplatnily pro praktické využití.

Přestože umíme nalézt velmi kvalitní plány pohotovostní služby na dané sadě incidentů, lze práci do budoucna rozšířit hned o několik věcí.
Především by bylo zajímavé více prozkoumat, jak vypadají optimální plány, které by odbavovaly jen o něco méně incidentů, ale byly by výrazně levnější.
Takové plány by se daly hledat pomocí vhodně zvolené účelové funkce, která by poskytovala dobře nastavený kompromis mezi cenou a počtem odbavených incidentů. 

Dalším zajímavým rozšířením by bylo přidat incidentům typ, který by určoval, jaká záchranná vozidla a týmy by mohly incident odbavit.
Jednotlivé týmy nebo vozidla by byly různě drahé, právě podle toho, jaké typy incidentů by mohly odbavovat. 
Implementace takového rozšíření vyžaduje pouze úpravu simulace a přidání nových tahů u metod, které využívají sousedství.

V neposlední řadě by bylo zajímavé zkoumat, jak se nalezené optimální plány chovájí v krajních případech, kdy se například najednou stane velké množství incidentů
na jedné lokalitě, nebo na velmi odlehlé lokalitě, případně i ve stejný čas. Aby bylo možné takové plány hledat, je potřeba, aby plány uměly dobře generalizovat.
Metody by tak musely krajní případy uvažovat už v rámci hledání a předem alokovat vozidla a týmy, i když by většinu času nebyly potřeba.
To nás vede na další rozšíření, kdy by bylo dovoleno, aby týmy naalokované na jedné výjezdové stanici mohly pomáhat v krajních případech týmům na jiných stanicích.
Tomuto mechanismu se říká \textit{dynamic dispatch} a v krajních případech by se mohlo jednat o nejefektivnější způsob, jak zdroje rozmisťovat.

