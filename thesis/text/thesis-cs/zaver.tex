\chapter*{Závěr}
\addcontentsline{toc}{chapter}{Závěr}

Tato práce se zabývala problémem nalezení optimálního plánu pohotovostní služby. 
V první kapitole jsme problém zformalizovali a důkladně zanalyzovali, abychom zvolili vhodné metody řešení.
Ukázalo se, že nejvhodnější je problém modelovat jako optimalizační úlohu s jednou účelovou funkcí.
Účelová funkce je vhodným složením ceny plánu a počtu úspěšně odbavených incidentů plánu na dané sadě incidentů.
Abychom věrohodně zjistili počet úspěšně odbavených incidentů, spustili jsme diskrétní simulaci, která pro danou sadu incidentů simuluje chování plánu v průběhu jednoho dne.
Použití simulace omezilo možné techniky, které se běžně používají pro řešení těžkých optimalizačních problému, jako například lineární programování,
na druhou stranu, umožňuje věrohodně získat počet úspěšně odbavených incidentů, jako kdyby plán již skutečně incidenty odbavoval v terénu.
 
V druhé kapitole jsme diskutovali několik metod, od využití dynamického programování, až po použití metaheuristických přístupů.
Nalezli jsme rekurzivní vztah mezi plány odbavující o jedna méně incidentů,
a využili jej pro vyřešení úlohy pomocí dynamického programování. Přestože výpočetní složitost je exponenciální vůči počtu incidentů, 
tak se jedná o dramatické zlepšení, oproti naivnímu řešení, které jenom náhodně vytváří plány.

Ve třetí kapitole jsme nalezené metody aplikovali na konkrétní pohotovostní službu, a to sice na pohotovostní službu hlavního města Prahy.
Synteticky jsme vygenerovali sadu incidentů, která vhodně reprezentuje, jak skutečně by se incidenty v průběhu dne mohli odehrávat, podle volně dostupných dat přímo od pražské pohotovostní služby.
Při aplikaci metod jsme zjistili, že některé metody jsou vhodnější než jiné. Například se ukázalo, že nemá smysl používat tabu prohledávání, jelikož je obtížné nalézt i jedno lokální optimum.
Na druhou stranu, kombinace lokálního prohledávání nebo simulovaného žíhání spolu s dynamickým programováním se ukázali jako nejlepší metody pro praktické využití,
kdy nalezené plány uměly úspěšně odbavit všechny incidenty jen s polovinou vozidel a do stovky týmů záchranářů.
Nalezené plány jsou kvalitní, a jistě by se uplatnili pro praktické využití.

