%%% The main file. It contains definitions of basic parameters and includes all other parts.

%% Settings for single-side (simplex) printing
% Margins: left 40mm, right 25mm, top and bottom 25mm
% (but beware, LaTeX adds 1in implicitly)
\documentclass[12pt,a4paper]{report}
\setlength\textwidth{145mm}
\setlength\textheight{247mm}
\setlength\oddsidemargin{15mm}
\setlength\evensidemargin{15mm}
\setlength\topmargin{0mm}
\setlength\headsep{0mm}
\setlength\headheight{0mm}
% \openright makes the following text appear on a right-hand page
\let\openright=\clearpage

%% Settings for two-sided (duplex) printing
% \documentclass[12pt,a4paper,twoside,openright]{report}
% \setlength\textwidth{145mm}
% \setlength\textheight{247mm}
% \setlength\oddsidemargin{14.2mm}
% \setlength\evensidemargin{0mm}
% \setlength\topmargin{0mm}
% \setlength\headsep{0mm}
% \setlength\headheight{0mm}
% \let\openright=\cleardoublepage

%% Generate PDF/A-2u
\usepackage[a-2u]{pdfx}

%% Character encoding: usually latin2, cp1250 or utf8:
\usepackage[utf8]{inputenc}

%% Prefer Latin Modern fonts
\usepackage{lmodern}

%% added by me
\usepackage{parskip}
\usepackage{algorithmic}
\usepackage{algorithm}
%%

%% Further useful packages (included in most LaTeX distributions)
\usepackage{amsmath}        % extensions for typesetting of math
\usepackage{amsfonts}       % math fonts
\usepackage{amsthm}         % theorems, definitions, etc.
\usepackage{bbding}         % various symbols (squares, asterisks, scissors, ...)
\usepackage{bm}             % boldface symbols (\bm)
\usepackage{graphicx}       % embedding of pictures
\usepackage{fancyvrb}       % improved verbatim environment
\usepackage{natbib}         % citation style AUTHOR (YEAR), or AUTHOR [NUMBER]
\usepackage[nottoc]{tocbibind} % makes sure that bibliography and the lists
% of figures/tables are included in the table
% of contents
\usepackage{dcolumn}        % improved alignment of table columns
\usepackage{booktabs}       % improved horizontal lines in tables
\usepackage{paralist}       % improved enumerate and itemize
\usepackage{xcolor}         % typesetting in color

%%% Basic information on the thesis

% Thesis title in English (exactly as in the formal assignment)
\def\ThesisTitle{Emergency services shift plan optimization}

% Author of the thesis
\def\ThesisAuthor{Tomáš Arnold Tillmann}

% Year when the thesis is submitted
\def\YearSubmitted{2023}

% Name of the department or institute, where the work was officially assigned
% (according to the Organizational Structure of MFF UK in English,
% or a full name of a department outside MFF)
\def\Department{Department of Software Engineering}

% Is it a department (katedra), or an institute (ústav)?
\def\DeptType{Department}

% Thesis supervisor: name, surname and titles
\def\Supervisor{Adam Šmelko}

% Supervisor's department (again according to Organizational structure of MFF)
\def\SupervisorsDepartment{Department of Distributed and Dependable Systems }

% Study programme and specialization
\def\StudyProgramme{Programming and Software Development}
\def\StudyBranch{Computer science}

% An optional dedication: you can thank whomever you wish (your supervisor,
% consultant, a person who lent the software, etc.)
\def\Dedication{%
}

% Abstract (recommended length around 80-200 words; this is not a copy of your thesis assignment!)
\def\Abstract{%
}

% 3 to 5 keywords (recommended), each enclosed in curly braces
\def\Keywords{%
	{one} {two}
}

%% The hyperref package for clickable links in PDF and also for storing
%% metadata to PDF (including the table of contents).
%% Most settings are pre-set by the pdfx package.
\hypersetup{unicode}
\hypersetup{breaklinks=true}

% Definitions of macros (see description inside)
%%% Tento soubor obsahuje definice různých užitečných maker a prostředí %%%
%%% Další makra připisujte sem, ať nepřekáží v ostatních souborech.     %%%

%%% Přepínače podle typu práce

\def\TypeBc{bc}
\def\TypeMgr{mgr}
\def\TypePhD{phd}
\def\TypeRig{rig}

\ifx\ThesisType\TypeBc
\def\ThesisTypeName{bakalářská}
\def\ThesisTypeGenitive{bakalářské}
\def\ThesisTypeAccusative{bakalářskou}
\def\ThesisTypeTitle{BAKALÁŘSKÁ PRÁCE}
\fi

\ifx\ThesisType\TypeMgr
\def\ThesisTypeName{diplomová}
\def\ThesisTypeGenitive{diplomové}
\def\ThesisTypeAccusative{diplomovou}
\def\ThesisTypeTitle{DIPLOMOVÁ PRÁCE}
\fi

\ifx\ThesisType\TypePhD
\def\ThesisTypeName{disertační}
\def\ThesisTypeGenitive{disertační}
\def\ThesisTypeAccusative{disertační}
\def\ThesisTypeTitle{DISERTAČNÍ PRÁCE}
\fi

\ifx\ThesisType\TypeRig
\def\ThesisTypeName{rigorozní}
\def\ThesisTypeGenitive{rigorozní}
\def\ThesisTypeAccusative{rigorozní}
\def\ThesisTypeTitle{RIGOROZNÍ PRÁCE}
\fi

\ifx\ThesisTypeName\undefined
\PackageError{thesis}{Neznámý typ práce.}{Zkontrolujte prosím definici makra ThesisType v souboru metadata.tex.}
\fi

%%% Drobné úpravy stylu

% Tato makra přesvědčují mírně ošklivým trikem LaTeX, aby hlavičky kapitol
% sázel příčetněji a nevynechával nad nimi spoustu místa. Směle ignorujte.
\makeatletter
\def\@makechapterhead#1{
  {\parindent \z@ \raggedright \normalfont
   \Huge\bfseries \thechapter\quad #1
   \par\nobreak
   \vskip 20\p@
}}
\def\@makeschapterhead#1{
  {\parindent \z@ \raggedright \normalfont
   \Huge\bfseries #1
   \par\nobreak
   \vskip 20\p@
}}
\makeatother

% Toto makro definuje kapitolu, která není očíslovaná, ale je uvedena v obsahu.
\def\chapwithtoc#1{
\chapter*{#1}
\addcontentsline{toc}{chapter}{#1}
}

% Trochu volnější nastavení dělení slov, než je default.
\lefthyphenmin=2
\righthyphenmin=2

% Zapne černé "slimáky" na koncích řádků, které přetekly, abychom si
% jich lépe všimli.
%\overfullrule=1mm

%%% Makra pro definice, věty, tvrzení, příklady, ... (vyžaduje baliček amsthm)

\theoremstyle{plain}
\newtheorem{veta}{Věta}
\newtheorem{lemma}[veta]{Lemma}
\newtheorem{tvrz}[veta]{Tvrzení}
\newtheorem{definice}{Definice}

\theoremstyle{remark}
\newtheorem*{dusl}{Důsledek}
\newtheorem*{pozn}{Poznámka}
\newtheorem*{prikl}{Příklad}

% Alias pro prostředí proof s českým názvem
\newenvironment{dukaz}{\begin{proof}}{\end{proof}}

%%% Styl popisek obrázků

\ifcsname DeclareCaptionStyle\endcsname
\DeclareCaptionStyle{thesis}{style=base,font=small,labelfont=bf,labelsep=quad}
\captionsetup{style=thesis}
\captionsetup[algorithm]{style=thesis,singlelinecheck=off}
\captionsetup[listing]{style=thesis,singlelinecheck=off}
\fi

%%% Prostředí pro sazbu kódu, případně vstupu/výstupu počítačových
%%% programů.

\DefineVerbatimEnvironment{code}{Verbatim}{fontsize=\small, frame=single}

% Nastavení pro lstlisting -- výpisy programů se zvýrazňováním syntaxe
\ifcsname lstset\endcsname
\lstset{
  language=C++,
  tabsize=2,
  showstringspaces=false,
  basicstyle=\footnotesize\tt\color{black!75},
  identifierstyle=\bfseries\color{black},
  commentstyle=\color{green!50!black},
  stringstyle=\color{red!50!black},
  keywordstyle=\color{blue!75!black}}
\fi

% Plovoucí výpisy kódu, používá se stejně jako figure
\ifcsname DeclareNewFloatType\endcsname
\DeclareNewFloatType{listing}{}
\floatsetup[listing]{style=ruled}
\floatname{listing}{Program}
\fi

%%% Prostor reálných, resp. přirozených čísel
\newcommand{\R}{\mathbb{R}}
\newcommand{\N}{\mathbb{N}}

%%% Užitečné operátory pro statistiku a pravděpodobnost
\DeclareMathOperator{\pr}{\textsf{P}}
\DeclareMathOperator{\E}{\textsf{E}}
\DeclareMathOperator{\var}{\textrm{var}}
\DeclareMathOperator{\sd}{\textrm{sd}}

%%% Příkaz pro transpozici vektoru/matice
\newcommand{\T}[1]{#1^\top}

%%% Asymptotické "O"
\def\O{\mathcal{O}}

%%% Vychytávky pro matematiku
\newcommand{\goto}{\rightarrow}
\newcommand{\gotop}{\stackrel{P}{\longrightarrow}}
\newcommand{\maon}[1]{o(n^{#1})}
\newcommand{\abs}[1]{\left|{#1}\right|}
\newcommand{\dint}{\int_0^\tau\!\!\int_0^\tau}
\newcommand{\isqr}[1]{\frac{1}{\sqrt{#1}}}

%%% Vychytávky pro tabulky
\newcommand{\pulrad}[1]{\raisebox{1.5ex}[0pt]{#1}}
\newcommand{\mc}[1]{\multicolumn{1}{c}{#1}}

%%% Zvýraznění údajů, které je potřeba doplnit: před odevzdáním práce odstraňte :)
\newcommand{\xxx}[1]{\textcolor{black}{#1}}

%%% Detailní nastavení bibliografie

\ifx\citet\undefined\else

% Maximální počet autorů, které uvádíme u jednoho díla (více se uvede pomocí "et al.")
%\ExecuteBibliographyOptions{maxnames=2}
% Totéž nastavavení specificky pro citace pomocí \citet{...}
\ExecuteBibliographyOptions{maxcitenames=2}
% Totéž nastavavení specificky pro seznam literatury
%\ExecuteBibliographyOptions{maxbibnames=2}

% Zkracování křestních jmen autorů: "E. A. Poe" místo "Edgar Allan Poe"
%\ExecuteBibliographyOptions{giveninits}
% Totéž bez teček ("EA Poe")
%\ExecuteBibliographyOptions{terseinits}

% Pokud máte problémy s lámáním dlouhých položek bibliografie na řádky,
% zkuste tento režim:
%\ExecuteBibliographyOptions{block=ragged}

% Ve stylech, které se neřídí podle ISO 690, můžeme také prohodit jméno a příjmení:
%\DeclareNameAlias{default}{family-given}

% Ve stylech podle ISO 690 vypadá lépe sázet příjmení kapitálkami:
\let\familynameformat=\textsc

% Více jmen autorů chceme v citacích oddělovat čárkami
% (zatímco v bibliografii norma předepisuje středníky)
\DeclareDelimFormat[textcite]{multinamedelim}{\addcomma\space}
\DeclareDelimFormat[textcite]{finalnamedelim}{\space a~}

\fi


% Title page and various mandatory informational pages
\begin{document}
%%% Titulní strana práce a další povinné informační strany

%%% Nápisy na přední straně desek
%%% Pokud je práce ve slovenštině, desky mají být česky.

% Desky obvykle nesázíme, ale pokud je chcete přidat, změnte \iffalse na \iftrue
\iffalse

\pagestyle{empty}
\hypersetup{pageanchor=false}
\begin{center}

\large
Univerzita Karlova

\medskip

Matematicko-fyzikální fakulta

\vfill

{\huge\bf\ThesisTypeTitle}

\vfill

{\huge\bf\ThesisTitle\par}

\vfill
\vfill

\hbox to \hsize{\YearSubmitted\hfil \ThesisAuthor}

\end{center}

\newpage\openright
\setcounter{page}{1}

\fi

%%% Titulní strana práce
%%% Pokud je práce ve slovenštině, tato strana zůstává česky.

\pagestyle{empty}
\hypersetup{pageanchor=false}

\begin{center}

\centerline{\mbox{\includegraphics[width=166mm]{img/logo-cs.pdf}}}

\vspace{-8mm}
\vfill

{\bf\Large\ThesisTypeTitle}

\vfill

{\LARGE\ThesisAuthor}

\vspace{15mm}

{\LARGE\bfseries\ThesisTitle\par}

\vfill

\Department

\vfill

{
\centerline{\vbox{\halign{\hbox to 0.45\hsize{\hfil #}&\hskip 0.5em\parbox[t]{0.45\hsize}{\raggedright #}\cr
Vedoucí \ThesisTypeGenitive{} práce:&\Supervisor \cr
\ifx\ThesisType\TypeRig\else
\noalign{\vspace{2mm}}
Studijní program:&\StudyProgramme \cr
\fi
}}}}

\vfill

Praha \YearSubmitted

\end{center}

\newpage

%%% Strana s čestným prohlášením k práci
%%% Pokud je práce ve slovenštině, tato strana zůstává česky.

\openright
\hypersetup{pageanchor=true}
\vglue 0pt plus 1fill

\noindent
Prohlašuji, že jsem tuto \ThesisTypeAccusative{} práci vypracoval(a) samostatně a výhradně
s~použitím citovaných pramenů, literatury a dalších odborných zdrojů.
Beru na~vědomí, že se na moji práci vztahují práva a povinnosti vyplývající
ze zákona č. 121/2000 Sb., autorského zákona v~platném znění, zejména skutečnost,
že Univerzita Karlova má právo na~uzavření licenční smlouvy o~užití této
práce jako školního díla podle §60 odst. 1 autorského zákona.

\vspace{10mm}

\hbox{\hbox to 0.5\hsize{%
V \hbox to 6em{\dotfill} dne \hbox to 6em{\dotfill}
\hss}\hbox to 0.5\hsize{\dotfill\quad}}
\smallskip
\hbox{\hbox to 0.5\hsize{}\hbox to 0.5\hsize{\hfil Podpis autora\hfil}}

\vspace{20mm}
\newpage

%%% Poděkování

\openright

\noindent
\Dedication

\newpage

%%% Povinná informační strana práce

\openright
{\InfoPageFont

\vtop to 0.5\vsize{
\setlength\parindent{0mm}
\setlength\parskip{5mm}

Název práce:
\ThesisTitle

Autor:
\ThesisAuthor

\DeptType:
\Department

Vedoucí \ThesisTypeGenitive{} práce:
\Supervisor, \SupervisorsDepartment

Abstrakt:
\Abstract

Klíčová slova:
{\def\sep{\unskip, }\ThesisKeywords}

\vfil
}

\vtop to 0.49\vsize{
\setlength\parindent{0mm}
\setlength\parskip{5mm}

Title:
\ThesisTitleEN

Author:
\ThesisAuthor

\DeptTypeEN:
\DepartmentEN

Supervisor:
\Supervisor, \SupervisorsDepartmentEN

Abstract:
\AbstractEN

Keywords:
{\def\sep{\unskip, }\ThesisKeywordsEN}

\vfil
}

}

\newpage

%%% Další stránky budeme číslovat
\pagestyle{plain}


%%% A page with automatically generated table of contents of the bachelor thesis

\tableofcontents

%%% Each chapter is kept in a separate file
\chapter*{Introduction}
\addcontentsline{toc}{chapter}{Introduction}


\setlength{\parindent}{0pt}

\chapter{Introduction}

\section{Problem definition}

One of the primary challenges faced by emergency medical service providers
(EMS) revolves around determining the optimal availability schedule for their
ambulances at specific depots. The primary objective is to maximize the number
of successfully handled incidents while minimizing the operational costs
associated with the ambulances. This allocation of ambulances and their
availability is commonly referred to as a shift plan.

A shift plan entails assigning shifts and ambulances for a given time interval,
typically a single day. It serves as a representation of when ambulances are
available throughout the day.

Each shift represents a time interval during which an ambulance crew operates
the vehicle, rendering it available for emergency response. Ambulance crews may
possess specialized skills or equipment, such as the inclusion of a doctor or
experienced personnel. Similarly, different ambulances can vary in size or
possess advanced medical tools. To capture these distinctions, each ambulance
is assigned a specific type.

Ambulance types provide an abstraction of these varying scenarios and specify
the incidents they are equipped to handle. For instance, certain incidents may
require specific tools or a larger ambulance. Consequently, only select
ambulances may be suitable for handling such incidents. Ultimately, our primary
concern int this regard is determining whether a given ambulance can
effectively respond to a particular incident, based on the mapping between
ambulance types and incident types. Hence such abstraction is sufficient for
our purposes.

Throughout the day, only one shift can be assigned to each ambulance, and there
may be instances where no shift is assigned, rendering the ambulance
unavailable for the day and unable to participate in incident response.

A shift plan incurs a cost, which we aim to minimize. This cost is determined
by the duration of the shift and the associated cost of the ambulance type.

In evaluating the performance of a shift plan, we measure its effectiveness in
handling a set of incidents. This set represents the spatial and temporal
distribution of incidents that occur within a day. We can obtain such incident
sets through two main approaches.

The first approach involves leveraging historical data, which document the
occurrence of incidents in a given area over months or years. By uniformly
sampling a representative subset of this data, we can test the generated shift
plan's performance. This approach assumes that past incident patterns will
persist in the future. Historical data provide the most reliable means of
simulating incident occurrences accurately.

The second approach involves defining a distribution that generates incidents
based on prior knowledge. For instance, we may observe that incidents primarily
occur during early mornings, around lunchtime, or at transportation hubs due to
commuting patterns. Conversely, we may notice a lower incidence rate during
nighttime compared to the daytime.

While the first approach more accurately depicts the occurrence of incidents in
the area of interest, obtaining such historical data can be challenging and
time-consuming. Moreover, these data often remain confidential, making it
difficult to access them, for example from other EMS organizations.

Therefore, for the purposes of this thesis, we will adopt the second approach,
generating representative incident sets using a predefined distribution. This
approach offers greater flexibility, allowing us to experiment with different
distributions and simulate various scenarios, including unexpected or extreme
situations.

Finally, with all this information at our disposal, we can now shift our focus
towards evaluating the performance of the shift plan against a set of incident
sets. In order to measure performance, we will assess the ratio of successfully
handled incidents to the total number of incidents in the set. A higher number
of handled incidents and lower costs indicate a better shift plan. While this
approach is reasonable, we will take a slightly different route in this thesis.
Since this thesis is associated with the Computer-Aided Dispatch System
Organization Logis Solutions s. r. o. (Logis), which employs advanced solutions
for existing EMS, Logis has the advantage of having historical incident data
sets available. This leads us to examine the success rates achieved by each
incident set on a given day. As the proposed solutions in this thesis may be
utilized in the production software developed by Logis for their EMS clients,
it would be ideal to generate a shift plan that maintains or surpasses the
success rates indicated by the historical data, while concurrently reducing
costs. Therefore, the success rate will be provided for each incident set, and
our optimization will solely focus on minimizing the cost.

Introducing the concept of success rate provides a useful framework for
considering a shift plan as either valid or invalid. A shift plan is deemed
valid if it can meet the given success rate, indicating its ability to satisfy
the incident requirements. Conversely, a shift plan is considered invalid if it
fails to meet the specified success rate. It is worth noting that there may be
scenarios where no shift plan can handle the incidents to meet the desired
success rate, resulting in all shift plans being invalid. However, this
particular case is not our primary concern, as the success rates derived from
historical data are based on shift plans employed on previous days, which were
not necessarily designed using highly sophisticated methods. Therefore, the
probability of encountering such a situation in high-quality historical data is
very low. Nonetheless, even in such scenario the proposed solutions will
prioritize maximizing the success rate to the greatest extent possible.

\section{Evaluating success rate of a shift plan}

Our primary objective is to ensure that the evaluation of the shift plan
closely reflects real-world conditions. It is essential that the success rate
of a shift plan, as determined by a given incident set and specific ambulances,
closely aligns with its real-world performance. To achieve this, we will employ
evaluation strategies that mirror those used in actual ambulance dispatching.

The most direct and flexible approach is to develop a simulation that emulates
the real-world dispatching of ambulances to incidents based on the shift plan.
This simulation will also incorporate the handling of incidents at the
occurrence location, as well as the travel to the hospital and return to the
depot. In reality, ambulance dispatching decisions are guided by a set of
rules. While these rules can become quite complex, often requiring approval
from a dispatcher, we can devise our own rules that closely approximate
real-world dispatching practices. By doing so, we can ensure that the
probabilities of dispatching in the simulation closely mirror those in the real
world and are easier to work with and simulate. Leveraging Logis's extensive
expertise as one of the top CAD dispatchers in the market, Logis has derived
the following abstracted rules. If multiple ambulances satisfy a given rule,
the selection process will follow the subsequent rules in the specified order.

\begin{enumerate}[(1)]
    \item Rule of Availability: Give priority to ambulances that are free, either
          stationed at the depot or returning from a hospital after completing an
          incident.
    \item Rule of Promptness: Prioritize the ambulance with the earliest estimated
          arrival time to the incident scene.
    \item Rule of Workload Distribution: Favor ambulances with the least overall active
          time, ensuring a fair distribution of workload among all available ambulances.
    \item Rule of Cost: If multiple ambulances can handle an incident equally well,
          choose the one with the lowest operating cost.
    \item If still more than one ambulance satisfies all the above rules, just choose one
          at random. In real life, this will be dispatcher's choice.
\end{enumerate}

In addition to the previously mentioned rules, before selecting the best
ambulance for an incident, we must ensure that the ambulance satisfies a set of
conditions. These conditions determine whether a particular ambulance is
capable of handling the incident. The conditions are as follows:

\begin{enumerate}[(1)]
    \item Time for Response: The ambulance must have sufficient time to travel from its
          current location to the incident within the maximum response time specified for
          the incident type.
    \item Duration at Incident: The ambulance must have enough time available to remain
          at the incident location for the expected duration, which is also determined by
          the incident type.
    \item Travel to Hospital: Sufficient time must be available for the ambulance to
          travel from the incident location to the nearest hospital.
    \item Patient Delivery: The ambulance must have enough time to transport the patient
          to the hospital, considering the specific duration required based on the
          incident type.
\end{enumerate}

It is crucial to ensure that the assigned ambulance can complete all the
necessary tasks outlined above within the allocated shift interval. By
verifying these conditions, we can determine the suitability of an ambulance
for a given incident, ensuring that it has the required time to carry out its
responsibilities effectively.

It can happen, that no ambulances can handle given incident based on above
conditions. In such a case, the incident is declared as unhandled, and the
success rate ration must be updated appropriately. This incident is therefore
ignored and no action is done.

Once the best ambulance is selected, the incident is assigned to that
ambulance, and the ambulance proceeds with handling the incident.

In the simulation, all incidents from the incidents set are processed
sequentially, one by one, and the best ambulance is selected solely based on
the above rules and conditions. Evaluating the incidents one by one closely
resembles the functioning of a shift plan in the real world, not exploiting the
bigger picture but only deciding on local information.

The accuracy and quality of the success rate obtained from the simulation are
heavily influenced by the calculated durations for various stages of incident
handling, including ambulance response time, incident handling at the scene,
patient transfer to the hospital, handling of the incident at the hospital, and
the estimated time required to return to the depot. To accurately calculate
these durations, a reliable distance calculator is essential.

One approach to achieve this is by discretizing the area of interest and
precalculating the distances for all pairs of locations, storing them in a
distance matrix. When a distance request is made, the distance calculator
simply retrieves the precalculated distance for the requested locations. The
time required can then be calculated using the velocity of the ambulance.
However, it is important to note that this approach does not consider factors
such as traffic or the type of route taken. A more appropriate approach might
be to precalculate not the estimated distance but the estimated duration, which
also eliminates the need for the ambulance's velocity information. To obtain
reasonable precalculated duration, Here or Google map services might be great
candidates. These durations can also be recalculated to better incorporate
these factors, such as traffic condition or road closure etc ...

Another factor to account for which might negatively influence the quality of
yielded success rate is that the variability of durations in real-life
scenarios vary. Factors such as traffic conditions, road closures, and
unexpected events can cause variations in ambulance response times, incident
handling durations, and patient transfer times. Therefore, because these
durations may vary in the real world, introducing stochastic factors to
simulate their variability could lead to improvements in estimating the success
rate. Despite that, given the necessity of running the simulation multiple
times, it is more crucial to ensure that the simulation effectively captures
real-life scenarios. Introducing additional stochastic elements may also
introduce unnecessary unpredictability to the simulation. Therefore, for the
sake of efficiency and reliability, the simulation does not incorporate such
stochastic factors.

\section{Problem formalization}
As the problem above is described, it is clear we have to find the most optimal
shift plan $s_o$ from set of all possible shift plans $S$, against given
incident sets $I$. Most optimal means it has to be valid,

\[
    \text{successRate}(s_o, I) \geq
    \text{givenSuccessRate}(I),
\]

and has the least possible cost over all valid shift plans $S_v$,

\[
    \forall s_v \in S_v: \text{cost}(s_o) \leq \text{cost}(s_v),
\]

where set $S_v$ of all valid shift plans is defined,

\[
    S_v = \{ s_v \in S \mid \text{successRate}(s_v, I) \geq \text{givenSuccessRate}(I) \}.
\]

There are two important considerations to note. First, this definition of an
optimal shift plan allows for the existence of multiple solutions that achieve
optimality. Therefore, we can refer to the set of all these optimal solutions
as $S_o$. Second, according to this definition, if no valid shift plans exist,
there is no well-defined optimum. While this may initially seem peculiar, it is
important to emphasize that in our problem formulation, we assume that it is
highly unlikely that such situation will occur, and they are not relevant to
our primary focus and it will only make proposed solutions unclear.

However, for the sake of consistency, we will provide a definition for the
optimum even in the unlikely event of no valid shift plans. It is worth noting,
though, that the subsequent proposed solutions will not consider this
particular case.

If there doesn't exist any shift plan in $S^I$ which satisfies given success
rate, the most optimal is then the shift plan with the highest success rate and
at the same time having the least cost. Let denote

\[
    S_r = \{s_r \in S \mid \text{successRate}(s_r, I) \geq \text{successRate}(s, I), \forall s \in S^T\}
\]

as a set of all shift plans with the highest success rate. The optimal $s_o \in
    S_r$ is now defined as:

\[
    \forall s_r \in S_r: \text{cost}(s_o, I) \leq \text{cost}(s_r, I).
\]

\newpage

\section{Exhaustive solution}

The most straightforward algorithms for finding the most optimal solution is to
employ a brute force approach by exhaustively enumerating all possible shift
plans. The algorithm initializes a global best shift plan at random or as first
one in the list and updates it in each iteration, comparing it to the current
shift plan being evaluated. By iterating through every possible shift plan, we
can ensure that we have identified the most optimal solution.

\begin{algorithmic}
    \STATE \textbf{Algorithm} ExhaustiveSolution
    \STATE \textbf{Input:} $I$, givenSuccessRate($I$)
    \STATE \textbf{Output:} $s_o$

    \STATE $s_o \leftarrow s_0 \in S^T$

    \FOR{$i \in \{ 1, 2, \ldots, |S^T| \}$}
    \IF{$\text{successRate}(s_i, I) \geq \text{givenSuccessRate}(I) \: \text{\AND} \text{cost}(s_i, I) \leq \text{cost}(s_o, I)$}
    \STATE $s_o \leftarrow s_i$
    \ENDIF
    \ENDFOR

    \STATE \textbf{Return} $s_o$
\end{algorithmic}

Although the brute force algorithm provides a correct solution, it suffers from
a significant drawback of being very slow. The runtime of this algorithm is
exponential, as the size of the set $S$ grows exponentially. Specifically, we
have:

\[
    |S| = |T|^{|A|},
\]

where $S$ represents the set of all allowed time intervals and $A$ represents
the set of all ambulances. Since the exhaustive algorithm needs to iterate
through all possible shift plans, the time complexity is:

\[
    \Theta(|S|).
\]

While it is reassuring to have a correct solution, the exponential runtime is
undesirable. We need to explore alternative approaches to improve the
efficiency of finding the optimal solution.
\chapter{Title of the second chapter}

\section{Title of the first subchapter of the second chapter}

\section{Title of the second subchapter of the second chapter}


\chapter*{Conclusion}
\addcontentsline{toc}{chapter}{Conclusion}


%%% Bibliography
%%% Bibliography (literature used as a source)
%%%
%%% We employ bibTeX to construct the bibliography. It processes
%%% citations in the text (e.g., the \cite{...} macro) and looks up
%%% relevant entries in the bibliography.bib file.
%%%
%%% The \bibliographystyle command selects, which style will be used
%%% for references from the text. The argument in curly brackets is
%%% the name of the corresponding style file (*.bst). Both styles
%%% mentioned in this template are included in LaTeX distributions.

\bibliographystyle{plainnat}    %% Author (year)
% \bibliographystyle{unsrt}     %% [number]

\renewcommand{\bibname}{Bibliography}

%%% Generate the bibliography. Beware that if you cited no works,
%%% the empty list will be omitted completely.

\bibliography{bibliography}

%%% If case you prefer to write the bibliography manually (without bibTeX),
%%% you can use the following. Please follow the ISO 690 standard and
%%% citation conventions of your field of research.

% \begin{thebibliography}{99}
%
% \bibitem{lamport94}
%   {\sc Lamport,} Leslie.
%   \emph{\LaTeX: A Document Preparation System}.
%   2nd edition.
%   Massachusetts: Addison Wesley, 1994.
%   ISBN 0-201-52983-1.
%
% \end{thebibliography}


%%% Figures used in the thesis (consider if this is needed)
\listoffigures

%%% Tables used in the thesis (consider if this is needed)
%%% In mathematical theses, it could be better to move the list of tables to the beginning of the thesis.
\listoftables

%%% Abbreviations used in the thesis, if any, including their explanation
%%% In mathematical theses, it could be better to move the list of abbreviations to the beginning of the thesis.
\chapwithtoc{List of Abbreviations}

%%% Attachments to the bachelor thesis, if any. Each attachment must be
%%% referred to at least once from the text of the thesis. Attachments
%%% are numbered.
%%%
%%% The printed version should preferably contain attachments, which can be
%%% read (additional tables and charts, supplementary text, examples of
%%% program output, etc.). The electronic version is more suited for attachments
%%% which will likely be used in an electronic form rather than read (program
%%% source code, data files, interactive charts, etc.). Electronic attachments
%%% should be uploaded to SIS and optionally also included in the thesis on a~CD/DVD.
%%% Allowed file formats are specified in provision of the rector no. 72/2017.
\appendix
\chapter{Attachments}

\section{First Attachment}

\openright
\end{document}