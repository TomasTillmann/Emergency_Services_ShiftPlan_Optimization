\setlength{\parindent}{0pt}

\chapter{Introduction}

\section{Problem definition}
One of the primary challenges faced by emergency medical service providers
(EMS) revolves around determining the optimal availability schedule for their
ambulances at specific depots. The primary objective is to maximize the number
of successfully handled incidents while minimizing the operational costs
associated with the ambulances. This allocation of ambulances and their
availability is commonly referred to as a shift plan.

A shift plan entails assigning shifts and ambulances for a given time interval,
typically a single day. It serves as a representation of when ambulances are
available throughout the day.

Each shift represents a time interval during which an ambulance crew operates
the vehicle, rendering it available for emergency response. Ambulance crews may
possess specialized skills or equipment, such as the inclusion of a doctor or
experienced personnel. Similarly, different ambulances can vary in size or
possess advanced medical tools. To capture these distinctions, each ambulance
is assigned a specific type.

Ambulance types provide an abstraction of these varying scenarios and specify
the incidents they are equipped to handle. For instance, certain incidents may
require specific tools or a larger ambulance. Consequently, only select
ambulances may be suitable for handling such incidents.

Ultimately, our primary concern is determining whether a given ambulance can
effectively respond to a particular incident, based on the mapping between
ambulance types and incident requirements. Throughout the day, only one shift
can be assigned to each ambulance, and there may be instances where no shift is
assigned, rendering the ambulance unavailable for the day and unable to
participate in incident response.

A shift plan incurs a cost, which we aim to minimize. This cost is determined
by the duration of the shift and the associated cost of the ambulance type.

In evaluating the performance of a shift plan, we measure its effectiveness in
handling a set of incidents. This set represents the spatial and temporal
distribution of incidents that occur within a day. We can obtain such incident
sets through two main approaches.

The first approach involves leveraging historical data, which document the
occurrence of incidents in a given area over months or years. By uniformly
sampling a representative subset of this data, we can test the generated shift
plan's performance. This approach assumes that past incident patterns will
persist in the future. Historical data provide the most reliable means of
simulating incident occurrences accurately.

The second approach involves defining a distribution that generates incidents
based on prior knowledge. For instance, we may observe that incidents primarily
occur during early mornings, around lunchtime, or at transportation hubs due to
commuting patterns. Conversely, we may notice a lower incidence rate during
nighttime compared to the daytime.

While the first approach more accurately depicts the occurrence of incidents in
the area of interest, obtaining such historical data can be challenging and
time-consuming. Moreover, these data often remain confidential, making it
difficult to access them, especially from other EMS organizations.

Therefore, for the purposes of this thesis, we will adopt the second approach,
generating representative incident sets using a predefined distribution. This
approach offers greater flexibility, allowing us to experiment with different
distributions and simulate various scenarios, including unexpected or extreme
situations.

Finaly having all this information at our display, we can focus on how to
evaluate performance of shift plan against a set of incidents sets.