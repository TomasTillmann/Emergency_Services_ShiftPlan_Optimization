\setlength{\parindent}{0pt}

\chapter{Introduction}

\section{Problem definition}
One of the primary challenges faced by emergency medical service providers
(EMS) revolves around determining the optimal availability schedule for their
ambulances at specific depots. The primary objective is to maximize the number
of successfully handled incidents while minimizing the operational costs
associated with the ambulances. This allocation of ambulances and their
availability is commonly referred to as a shift plan.

A shift plan entails assigning shifts and ambulances for a given time interval,
typically a single day. It serves as a representation of when ambulances are
available throughout the day.

Each shift represents a time interval during which an ambulance crew operates
the vehicle, rendering it available for emergency response. Ambulance crews may
possess specialized skills or equipment, such as the inclusion of a doctor or
experienced personnel. Similarly, different ambulances can vary in size or
possess advanced medical tools. To capture these distinctions, each ambulance
is assigned a specific type.

Ambulance types provide an abstraction of these varying scenarios and specify
the incidents they are equipped to handle. For instance, certain incidents may
require specific tools or a larger ambulance. Consequently, only select
ambulances may be suitable for handling such incidents.

Ultimately, our primary concern is determining whether a given ambulance can
effectively respond to a particular incident, based on the mapping between
ambulance types and incident requirements, hence such abstraction is sufficient
for our purposes. Throughout the day, only one shift can be assigned to each
ambulance, and there may be instances where no shift is assigned, rendering the
ambulance unavailable for the day and unable to participate in incident
response.

A shift plan incurs a cost, which we aim to minimize. This cost is determined
by the duration of the shift and the associated cost of the ambulance type.

In evaluating the performance of a shift plan, we measure its effectiveness in
handling a set of incidents. This set represents the spatial and temporal
distribution of incidents that occur within a day. We can obtain such incident
sets through two main approaches.

The first approach involves leveraging historical data, which document the
occurrence of incidents in a given area over months or years. By uniformly
sampling a representative subset of this data, we can test the generated shift
plan's performance. This approach assumes that past incident patterns will
persist in the future. Historical data provide the most reliable means of
simulating incident occurrences accurately.

The second approach involves defining a distribution that generates incidents
based on prior knowledge. For instance, we may observe that incidents primarily
occur during early mornings, around lunchtime, or at transportation hubs due to
commuting patterns. Conversely, we may notice a lower incidence rate during
nighttime compared to the daytime.

While the first approach more accurately depicts the occurrence of incidents in
the area of interest, obtaining such historical data can be challenging and
time-consuming. Moreover, these data often remain confidential, making it
difficult to access them, especially from other EMS organizations.

Therefore, for the purposes of this thesis, we will adopt the second approach,
generating representative incident sets using a predefined distribution. This
approach offers greater flexibility, allowing us to experiment with different
distributions and simulate various scenarios, including unexpected or extreme
situations.

Finally, with all this information at our disposal, we can now shift our focus
towards evaluating the performance of the shift plan against a set of incident
sets. In order to measure performance, we will assess the ratio of successfully
handled incidents to the total number of incidents in the set. A higher number
of handled incidents and lower costs indicate a better shift plan. While this
approach is reasonable, we will take a slightly different route in this thesis.
Since this thesis is associated with the Computer-Aided Dispatch System
Organization Logis Solutions s. r. o. (Logis), which employs advanced solutions
for existing EMS, Logis has the advantage of having historical incident data
sets available. This leads us to examine the success rates achieved by each
incident set on a given day. As the proposed solutions in this thesis may be
utilized in the production software developed by Logis for their EMS clients,
it would be ideal to generate a shift plan that maintains or surpasses the
success rates indicated by the historical data, while concurrently reducing
costs. Therefore, the success rate will be provided for each incident set, and
our optimization will solely focus on minimizing the cost.

Introducing the concept of success rate provides a useful framework for
considering a shift plan as either valid or invalid. A shift plan is deemed
valid if it can meet the given success rate, indicating its ability to satisfy
the incident requirements. Conversely, a shift plan is considered invalid if it
fails to meet the specified success rate. It is worth noting that there may be
scenarios where no shift plan can handle the incidents to meet the desired
success rate, resulting in all shift plans being invalid. However, this
particular case is not our primary concern, as the success rates derived from
historical data are based on shift plans employed on previous days, which were
not necessarily designed using highly sophisticated methods. Therefore, the
probability of encountering such a situation in high-quality historical data is
very low. Nonetheless, even in such scenario the proposed solutions will
prioritize maximizing the success rate to the greatest extent possible.