\chapter*{Úvod}
\addcontentsline{toc}{chapter}{Úvod}

Zásadním úkol, se kterým se potýkájí všechny pohotovostní služby, je plánování konkrétních směn.
Hlavním cílem tohoto plánování je rozdělení a rozvržení týmů záchranářů a dostupných záchranných vozidel na výjezdové stanice tak, aby byly jednotlivé incidenty efektivně odbaveny. 
Pod tím si můžeme představit, že během daného časového úseku, například jednoho dne, chceme úspěšně odbavit co největší počet incidentů, a to při nejnižsích možných vynaložených nákladech.
Nalézt takový optimální a efektivní plán je poměrně komplexní úkol. Zejména pak pro pohotovostní služby
vlastnící vyšší stovky záchranářů a vozidel, které jsou velmi časté například ve Spojených státech amerických. 
Nalézt takový plán je ale velmi žádoucí, protože efektivní využití zdrojů může vést až k několikanásobně úspěšnějšímu
počtu odbavených incidentů za nižší potřebné vynaložené náklady.

Cílem této práce je představit, aplikovat a porovnat metody, které nám pomohou taková optimální naplánování směn nalézt. 
Práce je rozdělena do tří částí.
V první části je důkladně popsán a zformalizován řešený problém jako optimalizační úloha a jakým způsobem namodelujeme pohotovostní službu a sadu incidentů.
Dále si popíšeme, jak konkrétně vyhodnotíme, kolik incidentů plán úspěšně odbaví.
Budeme zkoumat chování plánu pohotovostní služby na dané sadě incidentů, pomocí deterministické diskrétní simulace.
Také si zaanalyzujeme řešený problém, a o nalezené poznatky se budeme opírat při nalézání řešících metod, které jsou diskutovány v druhé části práce.

V druhé části se zabýváme několika možnými řešeními optimalizační úlohy.
První se pokusíme nalézt rychlá řešení pomocí dynamického programování.
Ukáže se, že jsme schopni nalézt rekurzivní vztahy mezi většími a menšími podproblémy, avšak v nejhorším případě se stále nebude jednat o řešení v polynomiálním čase.
Na další metody, které se podíváme budou metaheuristiké přístupy, které se pokusí nalézt alespoň dostatečně dobrá řešení za rozumnou dobu.
Podíváme se na nejčastěji používané metaheuristiky, které se běžně používají pro řešení náročných kombinatorických úloh.

V třetí části jednotlivé metody aplikujeme na reálnou pohotovostní službu, a to konkrétně na pohotovostní službu hlavního města Prahy.
Budeme zkoumat nalezené optimální plány jednotlivými metodami a následně je mezi sebou porovnáme.
Na závěr na základě dat vyhodnotíme, které metody jsou nejlepšími pro nalezení optimálních plánů, a které jsou doporučené pro reálné využití.
Tím bude naplněn cíl práce, kterým bylo nalézt nejvhodnější metodu, řešící problém nalezení optimálního pohotovostního plánu. 


