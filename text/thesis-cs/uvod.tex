\chapter*{Úvod}
\addcontentsline{toc}{chapter}{Úvod}

Základním úkolem, se kterým se potýkají všechny pohotovostní služby, je plánování konkrétních směn.
Hlavním cílem tohoto plánování je rozdělení a rozvržení týmů záchranářů a dostupných záchranných vozidel na
výjezdové stanice tak, aby byly jednotlivé incidenty efektivně odbaveny.

Pod tím si můžeme představit, že během daného časového úseku, například jednoho dne, chceme úspěšně odbavit co největší počet incidentů, a to při nejnižších
možných vynaložených nákladech.

Nalézt takový optimální a efektivní plán je poměrně komplexní úkol. Zejména pak pro pohotovostní služby
disponující vyššími stovkami záchranářů a vozidel, které jsou velmi časté například ve Spojených státech amerických.
Nalézt takový plán je ale velmi žádoucí, protože efektivní využití zdrojů může vést až k několikanásobně úspěšnějšímu
počtu odbavených incidentů za nižší potřebné, vynaložené náklady.

Cílem této práce je představit, aplikovat a porovnat metody, které nám pomohou taková optimální naplánování směn nalézt.
Práce je rozdělena do tří částí.

V první části je důkladně popsán a zformalizován řešený problém jako optimalizační úloha.
Věnujeme se v ní způsobu, kterým je pohotovostní služba, včetně dané sady incidentů, namodelována.
Dále si popíšeme, jak lze vyhodnotit efektivnost konkrétního plánu, a to na základě počtu úspěšně odbavených incidentů a jeho ceny.
Chování jednotlivých plánu pohotovostní služby budeme zkoumat na dané sadě incidentů, pomocí deterministické diskrétní simulace.
V této části práce si také podrobně zanalyzujeme řešený problém.
O tyto poznatky se budeme opírat při nalézání řešících metod, které jsou diskutovány v druhé části práce.

Druhá část práce představuje a popisuje několik možných řešení pro optimalizační úlohy.
Nejprve se pokusíme nalézt rychlá řešení pomocí dynamického programování.
Pozornost je věnována zejména rekurzivním vztahům mezi většími a menšími podproblémy.
Další metody, na které se podíváme budou metaheuristiké přístupy.
Podíváme se na nejčastěji používané metaheuristiky, které se běžně používají pro řešení náročných kombinatorických úloh.

Ve třetí části práce aplikujeme jednotlivé metody na reálnou případovou studii pohotovostní služby, a to konkrétně na pohotovostní službu hlavního města Prahy.
Budeme zkoumat plány, které jsou navrhnuty jako optimální jednotlivými metodami,
a následně je mezi sebou porovnáme. Na základě poznatků a dat získaných v předchozích částech práce pak vyhodnotíme,
které metody jsou nejlepšími pro nalezení optimálních plánů, a které bychom mohli i doporučit pro využití v reálné světě, čímž naplníme cíl této práce. 
