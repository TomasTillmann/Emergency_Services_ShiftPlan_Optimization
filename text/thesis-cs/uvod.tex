\chapter*{Úvod}
\addcontentsline{toc}{chapter}{Úvod}

V případě vážné nehody může být i několik minut rozdíl mezi stabilizováním pacienta nebo pacientem v kritickém stavu.
Z toho důvodu je velmi důležité plné pokrytí oblasti působnosti pohotovostní služby, ideálně zajištění pomoci do určitého času, nehledě na lokaci nebo aktuální vytíženosti.
Zároveň je žádoucí, aby zaměstanci a vozidla pohotovostních služeb byli využiti efektivně.
Chtěli bychom co nejméně zaměstanců, kteří budou naplánováni na určitý časový interval, aby byli připraveni a k dispozici v případě incidentu,
ale nakonec nebudou potřeba na výjezd, jelikož jsou nadbyteční a všechny incidenty byli vyřešeny kolegy.
Pro zajištění takových požadavků je nutné si velmi dobře rozmyslet plán pohotovostní služby.
Plán, který balancuje mezi co největším pokrytím a je připraven i na krajní případy a také mezí efektivním naplánováním pracovníku aby byli co nejvíce využiti.
K vymyšlení takového plánu je dobré vycházet z historických dat.
Všimnout si, v jakých lokalitách a v jakých časech se incidenty dějí a při rozhodování se na jaké sanitky v jakých časech budou
zaměstnanci naplánovaní brát tato data v úvahu.
Pohotovostní služby mají ovšem stovky sanitek se stovky zaměstnanci, a vymyslet optimální plán se může stát velmi rychle poměrně náročnou úlohou.
Přitom umět takový optimální plán najít je velmi žádoucí.
\\\\
To nás vede na hlavní téma této práce, ve které se zabývám problematikou jak takový optimální plán najít a zkoumám metody,
které umí v rozumném čase najít jeden, nebo i více takových optimálních plánu.
Tyto metody zkoumám jak z hlediska kvality nalezených plánu, tak i jak rychle je metoda umí nalézt.
Dále zkoumám jak se metody chovají v různých scénářích.
Od normálního dne, kdy dle historických dat se incidenty dějí podle předpokládání, až po kritické situace, které plán maximálně vytíží.
Tyto metody mezi sebou i porovnávám, a ukáži jaké metody patří mezi ty nejlepší a které bych doporučil pro použití.
\\\\
Data a statistiky potřebné pro zmíněnou analýzu získávám simulováním chodu pohotovostní služby s přiřazeným daným plánem, získaným některou z metodik.
Historické incidenty, podle kterých hledám optimální plány jsou syntenticky vytvořeny.
To stejné platí pro rozmístění stanic se sanitkami a výběrem lokace působnosti pohotovostní služby.
Avšak snažil jsem se data vytvořit tak, aby co nejlépe odráželi reálný svět.
Simulace je deterministická a při výběru sanitky pro obsluhu incidentu jsou navrženy pravidla tak, jak by se mohla sanitka vybírat v reálné situaci.

