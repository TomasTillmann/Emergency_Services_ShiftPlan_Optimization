\chapter*{Úvod}
\addcontentsline{toc}{chapter}{Úvod}

Jedním z hlavních problému, které musí každá pohotovostní služba řešit, je jak v průběhu dne naplánovat týmy záchranařů a vozidla, kterými disponuje tak,
aby zajišťovala co nejlepší pokrytí území, na kterém působí, a zároveň byli zaměstnanci přiměřeně vytížení a pohotovostní služba neplýtvala zbytečně svými zdroji.
Nalézt takový optimální plán je poměrně náročný úkol, navíc pro pohotovostní služby vlastnící vyšší stovky záchranářů a vozidel, které jsou velmi častými například ve Spojených státech.
Přitom nalézt takový optimální plán je velmi žádoucí, protože efektivní využití zdrojů může vést až k několikanásobně úspěšnějšímu počtu odbavených incidentů
za nižší potřebné vynaložené náklady.

V této prácí se konkrétně budeme zabývat problematikou, jak takový optimální plán najít a zkoumat metody,
které umí v rozumném čase najít jeden, nebo i více takových optimálních plánu.
Tyto metody budou studovány nejen z hlediska kvality nalezených plánu, ale i jak rychle je metoda umí nalézt.
Jak takové optimální plány hledat lze několika způsoby. V této práci budeme především vycházet z historických dat.
Respektivě budeme zkoumat, kde a kdy se v minulosti na území děly incidenty, a podle toho budeme budovat takové plány, které budou na těchto incidentech fungovat efektivně.
Tyto metody na závěr práce aplikujeme na pohotovostní službu Prahy. Následně je porovnáma a ukážeme si, jaké metody patří mezi ty nejlepší a které jsou doporučené pro reálné využití. 

Data a statistiky potřebné pro zmíněnou analýzu získáme simulováním chodu pohotovostní služby s přiřazeným daným plánem, kde simulaci bude velmi věrohodně odpovídat chování plánu,
jako by skutečně byl takto vybrán do terénu.

