\chapter{Aplikace metod}

V této kapitole se budeme zabývat aplikováním a porovnáváním metod, kterým jsme se věnovali v kapitole \ref{chap:reseniOptUloh}.
Synteticky vygenerujeme data, která budou reprezentovat prostředky pohotovostní služby a sady incidentů. 
Data reprezentující prostředky pohotovostní služby budou modelovat nějakou reálnou pohotovostní službu.
V rámci této práce budeme modelovat Pražskou pohotovostní službu pro rok 2017, podle veřejně dostupných dat
poskytnutých přímo ze zdravotnické záchranné služby hlavního města Prahy.
\footnote{https://www.zzshmp.cz/wp-content/uploads/2017/12/Statistiky-160let-ZZSHMP.pdf}.
Chování nalezených optimálních plánů budeme následně zkoumat na různých sadách incidentů.

\section{Generování dat}


\section{Aplikace prohledávání plánů optimálními tahy}

\section{Aplikace lokálního prohledávání}

\section{Aplikace tabu prohledávání}

\section{Aplikace simulovaného žíhání}

\section{Porovnání metod}

