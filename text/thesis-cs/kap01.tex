\chapter{Optimalizační problém}

\section{Popis problému}

Pohotovostní služba vlastní pohotovostní vozidla, rozmístěná na výjezdových stanicích umístěných na území, které pohotovostní služba pokrývá.
Na každé výjezdové stanici působí několik týmu záchranářu. Ti jsou v případě pohotovosti připraveni obsloužit vozidlo na výjezdové stanici a vyjet na místo incidentu.
Plán pohotovostní služby je alokace záchranářů na konkrétní výjezdové služby v konkrétním časovém úseku v rámci jednoho dne.
Na výjezd je potřeba více záchranářu. Vždy je nutný řidič a kvalifikovaný lekář. Často je však i potřeba pomocné záchranáře.
Toto seskupení považujeme za tým záchranářů.
Tým záchranářů společně se záchraným vozidlem považujeme za záchranou jednotku.
\\\\
Pohotovostní plány mají různou kvalitu. Kvalita plánu se odvíjí od počtu incidentů, které byli úspěšně odbaveny a od efektivity využití záchranářů.
Incident je úspěšně odbaven, pokud v čase incidentu je k dispozici záchranná jednotka potřebná pro odbavení incidentu a zároveň jednotka dorazila včas na místo incidentu.
Incidenty máme různých typů, podle hlediska kritičnosti a podle potřeby výbavy a kvalifikace, kterou musí záchranná jednotka mít. 
Kritičtější incidenty definují kratší dobu, ve které musí záchranná jednotka dorazit na místo incidentu.
Stejně tak jsou incidenty, kde je na místě potřeba speciální výbava nebo kvalifikovanější tým záchranářu.
Pokud v čase incidentu neexistuje záchranná jednotka splňující tyto podmínky, tak incident považujeme za neúspěšně odbaven.
Efektivní využití záchranářu znamená, že záchranáři nejsou ani maximálně vytížení a ani nemají čas na odpočinek.
Maximální vytížení by mohlo vést k hůře odvedené práci a dlouhodobě až k jejich vyhoření.
Zároveň ale není žádoucí, aby záchranaři byli připraveni na výjezdové stanici a za celou dobu směny neměli ani jeden, nebo jen minimální počet výjezdů.
Taková situace vede k zbytečnému proplácení mezd záchranářu a údržby záchranných vozidel a pohotovostní službu tak stojí čas a peníze, které by mohla vynaložit efektivnějším způsobem.
V ideálním případě bychom chtěli, ať je efektivita záchranářu využita rovnoměrně napříč přes všechny výjezdové stanice, i když samozřejmě na některých oblastech dochází
k incidentům častěji a tedy nutně některé výjezdové stanice budou více vyčerpané než ostatní.
\\\\
Kvalita pohotovostního plánu se tedy odvíjí od počtu úspěšně odbavených incidentů a efektivity naplánování záchranářů.
Přirozeně vyvstává otázka, jakým způsobem najít ten nejkvalitnější plán, za účelem dosáhnutí co největší úspěšnosti za co nejméně zbytečně vynaložených zdrojů? 
Pro odpověď je první nutno problém formalizovat.

\clearpage

\section{Formalizace problému}

Mějme množinu všech možných plánů pohotovostních služeb $P$. Definujme funkci určující kvalitu plánu $q\colon P \rightarrow \mathbb{R}$.
Chceme najít takové $p* \in P$, pro které platí:

$$
q(p*) \geq q(p), \forall p \in P
$$
.
\\
\\
Jedná se tedy o maximalizační úlohu. Chceme najít pro jaký plán $p$ nabývá funkce $q$ globální maximum. Tak zjistíme plán s nejvyšší kvalitou.
Rozmysleme si, jak nadefinujeme plán. Plán je alokace týmů záchranářu na jednotlivé výjezdové stanice v určitý čas v rámci jednoho dne.
Můžeme na to nahlížet jako na přiřazení časových intervalů, směn, na výjezdové stanice, kde by každý interval odpovídal jednomu týmu záchranářů.
Nechť $S$ je množina výjezdových stanic, $I \subseteq (\mathbb{R^+}, \mathbb{R^+})$ množina všech časových intervalů a $I_C \subset I$ množina možných časových intervalů.
Například, $(8, 17) \in I_C$, značí směnu od 8 hodin do 17 hodin.
Dále nechť $T$ množina týmu, které nám jsou k dispozici. Definujme

$$
p_I \colon T \rightarrow I_C
$$

a

$$
p_S \colon T \rightarrow S
$$
.
\\
\\
Plán pohotovostní služby pak je dvojice $p = (p_I, p_S)$.
Všimněme si, že s takovou definicí můžeme na optimalizační problém nahlížet jako na dvě podoptimalizační úlohy.
První, optimálně namapovat týmy záchranářu na výjezdové stanice a druhý, optimálně každému týmu přiřadit směnu.
Může se stát, že některé týmu nebudeme chtít vůbec naalokovat, například z důvodu zvýšení efektivity plánu, pak je odebereme z $T$.

