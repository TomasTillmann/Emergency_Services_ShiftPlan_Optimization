\chapter{Optimalizační problém}

\section{Popis problému}

Pohotovostní služba vlastní pohotovostní vozidla, rozmístěná na výjezdových stanicích umístěných na území, které pohotovostní služba pokrývá.
Na každé výjezdové stanici působí několik záchranářu. Ti jsou v případě pohotovosti připraveni obsloužit vozidlo na výjezdové stanici a vyjet na místo incidentu.
Na výjezd je potřeba více záchranářu. Vždy je nutný řidič a kvalifikovaný lekář. Často je však i potřeba pomocné záchranáře.
Toto seskupení považujeme za tým záchranářů.
Tým záchranářů společně se záchraným vozidlem považujeme za záchranou jednotku.
Plán pohotovostní služby je alokace záchranných vozidel a týmů záchranářů na konkrétní výjezdové stanice a přiřazení pracovní služby každému týmu záchranářů v rámci jednoho dne.
\\\\
Pohotovostní plány mají různou kvalitu. Kvalita plánu se odvíjí od počtu incidentů, které byli úspěšně odbaveny a od efektivity využití záchranářů a záchranných vozidel.
Incident je úspěšně odbaven, pokud v čase incidentu je k dispozici záchranná jednotka potřebná pro odbavení incidentu a zároveň jednotka dorazila včas na místo incidentu.
Pokud v čase incidentu neexistuje záchranná jednotka splňující tyto podmínky, tak incident považujeme za neúspěšně odbaven.
Efektivní využití záchranářu znamená, že záchranáři nejsou ani maximálně vytížení, ale ani nemají čas na odpočinek.
Maximální vytížení by mohlo vést k hůře odvedené práci a dlouhodobě až k jejich vyhoření.
Zároveň ale není žádoucí, aby záchranaři byli připraveni na výjezdové stanici a za celou dobu směny neměli ani jeden, nebo jen minimální počet výjezdů.
Efektivní využití záchranných vozidel znamená, že vozidla nejsou zbytečně přepravována a udržována na výjezdových stanicích.
Taková situace vede k zbytečnému proplácení mezd záchranářu a údržby záchranných vozidel a pohotovostní službu tak stojí čas a peníze, které by mohla vynaložit efektivnějším způsobem.
V ideálním případě bychom chtěli, ať je efektivita záchranářu využita rovnoměrně napříč přes všechny výjezdové stanice, i když samozřejmě na některých oblastech dochází
k incidentům častěji a tedy nutně některé výjezdové stanice budou více vyčerpané než ostatní.
\\\\
Kvalita pohotovostního plánu se tedy odvíjí od počtu úspěšně odbavených incidentů, efektivity naplánování záchranářů a záchranných vozidel.
Přirozeně vyvstává otázka, jakým způsobem najít ten nejkvalitnější plán, za účelem dosáhnutí co největší úspěšnosti za co nejméně zbytečně vynaložených zdrojů? 
Pro odpověď je první nutno problém formalizovat.

\clearpage

\section{Formalizace problému}

Mějme množinu všech možných plánů pohotovostních služeb $P$. Definujme ohodnocovací funkci určující kvalitu plánu $q\colon P \rightarrow \mathbb{R}$.
Chceme najít takové $p* \in P$, pro které platí:

$$
q(p*) \geq q(p), \forall p \in P
$$
.
\\
\\
Jedná se tedy o maximalizační úlohu. Chceme najít pro jaký plán $p$ nabývá funkce $q$ globální maximum. Tak zjistíme plán s nejvyšší kvalitou.
Rozmysleme si, jak nadefinujeme plán. Plán je alokace týmů záchranářu a záchranných vozidel na jednotlivé výjezdové stanice, kde týmy záchranářů jsou k dispozici v určitý čas v rámci jednoho dne.
Můžeme na to nahlížet jako na přiřazení časových intervalů, směn, na výjezdové stanice, kde by každý interval odpovídal jednomu týmu záchranářů a na přiřazení záchranných vozidel,
které budou na výjezdových stanicích pro záchranáře k dispozici.
Nechť $S$ je množina výjezdových stanic, $I \subseteq (\mathbb{R^+}, \mathbb{R^+})$ množina všech časových intervalů a $I_C \subset I$ množina možných časových intervalů.
Například, $(8, 17) \in I_C$, značí směnu od 8 hodin do 17 hodin.
Dále nechť $T$ množina týmu, které nám jsou k dispozici a $A$ množina záchranných vozidel, které nám jsou k dispozici. Definujme

\begin{align}
p_I \colon T &\rightarrow I_C \\
p_S \colon T &\rightarrow S \\
p_A \colon A &\rightarrow S 
\end{align}
.
\\
\\
Plán pohotovostní služby pak je trojice

\begin{align}
p = (p_I, p_S, p_A)
\end{align}
.
\\
\\
Všimněme si, že s takovou definicí můžeme na optimalizační problém nahlížet jako na tři podoptimalizační úlohy.
První, optimálně namapovat týmy záchranářu na výjezdové stanice, druhý, optimálně každému týmu přiřadit směnu a třetí, optimálně alokovat záchranná vozidla na jednotlivé výjezdové stanice.
Může se stát, že některé týmy nebudeme chtít vůbec naalokovat, například z důvodu zvýšení efektivity plánu, pak je odebereme z $T$. To stejné platí pro záchranná vozidla, odebereme z $A$.

\clearpage

\section{Simulace}

Momentálně máme popsanou optimalizační úlohu, kterou řešíme. Ještě nám ale chybí jedna velmi podstatná část.
Tou je konkrétní způsob, definice ohodnocovací funkce $q$, kterým zjistíme kvalitu libovolného plánu pohotovostní služby.
Jak již bylo zmíněno, ta by se měla co nejvíce odrážet z úspěšnosti odbavených incidentů a efektivity alokace týmů záchranářů a záchranných vozidel.
To je pro 


