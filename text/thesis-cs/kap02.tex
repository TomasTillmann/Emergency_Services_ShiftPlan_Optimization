\chapter{Řešení optimalizační úlohy}\label{chap:reseniOptUloh}

V této kapitole vyzkoušíme různé metody, které můžeme pro nalezení optimálního plánu pohotovostní služby použít.

\section{Dynamické programování}\label{kap:dynamicProgram}

\textit{Dynamické programování} je technika řešení problému, která si průběžně ukládá řešení menších podúloh a pomocí rekurzivního vztahu menší podúlohy s větší,
definovaného \textit{Bellmanovou rovností}, řeší větší podúlohy efektivněji, až po vyřešení původní úlohy. 
Průběžnému ukládání výsledků podúloh se říká \textit{memoizace}.
Díky této memoizaci dynamické programování neprohledává prostor řešení duplicitně, a tak je často velmi efektivní metodou pro řešení optimalizačních úloh (\citet{dynamic}).
Triviálním příkladem využití dynamického programování je výpočet $n$-tého \textit{Fibbonaciho čísla}. (\citet{mares}, kap. 12).

Pro nás zajímavějším příkladem využití dynamického programování je řešení úlohy kombinatorické optimalizace \textit{Problému batohu}.

\begin{definice}[Problém batohu]
  Nechť $n$ počet předmětů, které chceme vložit do batohu s kapacitou $c$.
  Každý předmět $i$ má výdělek $p_i$ a váhu $w_i$. Problém batohu pak je,

  \begin{alignat*}{2}
    &\normalfont \text{maximalizuj} \hspace{30pt} &z = \sum_{i=1}^n p_i x_i \\
    \\
    &\normalfont \text{splňující} \hspace{30pt} &\sum_{i = 1}^{n} w_i x_i \leq c,
  \end{alignat*}
  \\
  kde $x_i = 1$, pokud předmět $i$ je v batohu, jinak $0$.
  \\
\end{definice}

Existuje $2^n$ možností, které předměty vložíme do batohu. Naivní řešení prohledání všech možností tak běží v čase $\mathcal{O} (2^n)$.
Avšak pomocí dynamického programování lze vyřešit problém batohu v \textit{pseudopolynomiálním} čase $\mathcal{O}(nc)$,
což pro velká $n$ a konstatní $c$ je až exponenciálním zlepšením.

Bellmanovy rovnice vyjadřující rekurzivní vztahy vypadají následovně,

\begin{definice}[Rekurzivní vztah pro problém batohu]\label{rov:KPrekurz}
  \begin{align*}
    m_{i, c'} &= m_{i - 1, c'} \text{ pokud } w_i > c', \\
    m_{i, c'} &= \max (m_{i - 1, c'}, m_{i - 1, c' - w_i} + p_i) \text{ pokud } w_i \leq c',
  \end{align*}
  \\
  pro $0 \leq c' \leq c$ aktuální uvažovaná kapacita batohu.
\end{definice}

Rekurzivní vztah \ref{rov:KPrekurz} nám pouze říká, že ze znalostí optimálního výběru předmětů podpbroblému $m_{i-1, c' - w_i}$ a podproblému $m_{i - 1, c'}$,
umíme v konstantím čase zjistit optimální výběr předmětů pro aktuální problém $c'$, tedy $m_{i, c'}$.
Buď předmět $i$ použijeme při výběru, a tak aktuální optimální výběr je roven $m_{i - 1, c' - w_i} + p_i$,
nebo předmět při výběru nepoužijeme, takže aktuální optimální výběr je optimální výběr podproblému, $m_{i - 1, c'}$. 

\begin{algorithm}[h]
  \begin{algorithmic}[1]
  \Function{KnapsackProblem}{$n$, $c$, $p_i$, $w_i$, $1 \leq i \leq n$}
    \State $m_{i, j}$ \gets 0, $0 \leq i \leq n$, $0 \leq j \leq c$
    \For{$1 \leq i \leq n$}
      \For{$1 \leq j \leq c$}
        \If{$w_i > j$}
          \State $m_{i, j}$ \gets $m_{i - 1, j}$
        \Else
          \State $m_{i, j}$ \gets $\max$ ($m_{i - 1, j}$, $m_{i - 1, j - w_i} + p_i$)
        \EndIf
      \EndFor
    \EndFor
    \State \Return $m_{n,c}$
  \EndFunction
  \end{algorithmic}
  \caption{Problém batohu}
  \label{KP}
\end{algorithm}

Algoritmus \ref{KP} vrací sumu hodnot optimálních předmětů přidaných do batohu.
Jaké konkrétní předměty přispěly do sumy lze snado zjistit zpětným následováním rekurzivního vztahu \ref{rov:KPrekurz}.

Rádi bychom našli nějaký podobný rekurzivní vztah v problému hledání optimálního plánu.
Nalezením rekurzivních vztahů by pak simulace $s$ nebyla černou skříňkou a byla by splněna nutná podmínka \ref{veta:simulaceJakoCernaSkrinka} pro šanci na polynomiální řešení.

\section{Prohledávání prostoru plánů tahy}

Nechť množina incidentů 
\begin{equation}
  I = \{ i_1, \dots , i_n \} \text{, kde } \forall j, k \in \{ 1, \dots n\} \colon T_I(i_j) \leq T_I(i_k),
\end{equation}
čili incidenty jsou seřazené podle času nastání.

Definujme množiny incidentů
\begin{equation}\label{df:INC}
  I_1, I_2, \dots, I_n \text{, kde } \forall k \in \{ 1, \dots, n \} \colon I_k = \{ i_1, \dots, i_k \}.
\end{equation}

Incidenty $I_k$ jsou tak incidenty $I_{k-1}$ po odebrání incidentu $i_k$, který se odehrál jako poslední.

Nechť množiny optimálních plánů
\begin{alignat}{2}
  & P^*_{1}, P^*_2, \dots, P^*_{n} \subset P_C \quad && \hspace{15pt} \text{optimální plány na množinách incidentů $I_{1}, I_2, \dots, I_n$.}
\end{alignat}

Nadále budeme zkoumat otázku, zda jsme schopni ze znalosti $p^*_{k-1} \in P^*_{k-1}$ nalézt optimální plán $p^*_k \in P^*_k$ na množině incidentů $I_k$, pro nějaká $k \in \{ 1, \dots, n \}$.
V první řadě si potřebujeme nadefinovat jak obecně z nějakého plánu $p_1$ získat plán $p_2$. 
\begin{definice}[Tah]\label{df:tah}
  Nadefinujme tah jako funkci $T \colon P_C \rightarrow P_C$.
\end{definice}

\begin{definice}[Inverzní tah]\label{df:tah}
  Nechť tah $T$. Inverzní tah $T^{-1}$ k tahu $T$ je tah, pro který platí

  \begin{align*}
    T(p) = p' \Leftrightarrow T^{-1}(p') = p.
  \end{align*}
\end{definice}


Inverzní tah obecně nemusí existovat. Ukažme si, jaké podmínky musí platit, aby existoval.

\begin{veta}[Podmínka existence inverzního tahu]\label{veta:inverzNutnost}
  Inverzní tah $T^{-1}$ k tahu $T$ existuje, pokud $\forall p_1, p_2 \in P_C \colon T(p_1) = T(p_2) \implies p_1 = p_2$.
  Čili tah $T$ je prostá funkce.

\end{veta}
\begin{dukaz}
  Inverzní tah $T^{-1}$ zkonstruujeme.
  Z předpokladu věty víme, že neexistuje plán $p'$ takový, že $T(p_1) = T(p_2) = p'$ pro různá $p_1, p_2 \in P_C$. 
  Proto můžeme inverzní tah definovat jako

  \begin{alignat*}{2}
    & T^{-1}(p') = p, \quad && \hspace{10pt} \text{takové, že $T(p) = p'$}.
  \end{alignat*}
\end{dukaz}

Tahy budeme používát pro převádění jednoho plánu na druhý nějakou posloupností tahů a budou se hodit pro hledání vztahů mezi plány.

\begin{definice}[Posloupnost tahů]
  Posloupností tahů velikosti $n$ rozumíme posloupnost tahů $T_1, T_2, \dots, T_n$.
  Řekneme, že z plánu $p$ získáme plán $p'$ posloupností tahů $T_1, T_2, \dots, T_n$,
  pokud $(T_1 \circ T_2 \circ \dots \circ T_n)(p) = p'$, kde symbol $\circ$ značí binární operaci skládání funkcí,
  kde se první vyhodnotí levá funkce a pak pravá.
\end{definice}

Ukažme si příklady nějakých tahů.

\begin{definice}[Kanonické tahy]
  Kanonickými tahy rozumíme tahy
  \begin{enumerate}
    \item
      Alokace týmu na výjezdovou stanici.

    \item
      Alokace záchranného vozidla na výjezdovou stanici.

    \item
      Prodloužení doby trvání směny již naalokovaného týmu záchranářu. 
  \end{enumerate}

  Formálně, kanonický tah alokace týmu $z'$ na výjezdovou stanici je funkce $M_{z'} : p \rightarrow p'$, kde
  \begin{align*}
    p_Z'(z) &=
      \begin{cases}
        v \neq v_{\emptyset} & \text{pro tým $z' \in Z \colon p_Z(z') = v_{\emptyset}, v \in V$}, \\
        p_Z(z) & \text{pro $\forall z \in Z \setminus \{ z' \}$},
      \end{cases}
      \\
    p_D'(z) &=
      \begin{cases}
        (d_1, d_2) \in D \colon d_2 - d_1 > 0 & \text{pro $z'$}, \\
        p_D(z) & \text{pro $\forall z \in Z \setminus \{ z' \}$},
      \end{cases}
      \\
    p_A'(a) &= p_A(a), \forall a \in A,
  \end{align*}

  pro $p_Z, p_D, p_A \in p$ a $p_Z', p_D', p_A' \in p'$, kde $p, p' \in P_C$ a $M_{z'}(p) = p'$.

  Kanonický tah alokace týmu $z'$ na výjezdovou stanici je funkce $M_{a'} : p \rightarrow p'$, kde
  \begin{align*}
    p_Z'(z) &= p_Z(z), \forall z \in Z,
    \\
    p_D'(z) &= p_D(z), \forall z \in Z,
    \\
    p_A'(a) &=
    \begin{cases}
      p_Z'(z) & \text{pro záchranné vozidlo $a' \in A \colon p_A(a') = v_{\emptyset}, v \in V$.}, \\
      p_A(a) & \text{pro $\forall a \in A \setminus \{ a' \}$},
    \end{cases}
  \end{align*}

  pro $p_Z, p_D, p_A \in p$ a $p_Z', p_D', p_A' \in p'$, kde $p, p' \in P_C$ a $M_{z'a'}(p) = p'$.

  Kanonický tah prodloužení směny týmu $z'$ je funkce $M_{z'd} : p \rightarrow p'$, kde
  \begin{align*}
    p_Z'(z) &= p_Z(z), \forall z \in Z \\
    p_D'(z) &=
      \begin{cases}
        (d'_1, d'_2) \in D \colon d'_2 > d_2 \land d'_1 = d_1  & \text{pro $z'$, kde $(d_1, d_2) = p_{D}(z)$} \\
        p_D(z) & \text{pro $\forall z \in Z \setminus \{ z' \}$},
      \end{cases}
      \\
    p_A'(a) &= p_A(a), \forall a \in A,
  \end{align*}

  pro $p_Z, p_D, p_A \in p$ a $p_Z', p_D', p_A' \in p'$, kde $p, p' \in P_C$ a $M_{z'd}(p) = p'$.
\end{definice}

Vidíme, že popsat tah formálně je komplikované a pro naše účely zbytečné.
Proto dále budeme tahy definovat slovně, ale zároveň tak, aby nedocházelo k nejasnostem.

\begin{veta}[Vztah mezi plány a kanonickými tahy]\label{veta:vztahMeziPlanyAKan}
  Každý plán $p \in P_C$ lze získat z prázdného plánu $p_0$ nějakou posloupností kanonických tahů.
  Prázdným plánem $p_0$ rozumíme plán, který nemá naalokován žádný tým ani vozidlo, čili $z \in Z, a \in A \colon p_Z(z) = p_A(a) = v_{\emptyset}$.
\end{veta}
\begin{dukaz}
  Nechť libovolný plán $p \in P_C$.
  Vystačíme si pouze s tahy alokace týmu a alokace záchranného vozidla.
  Pro $\forall z_i \in Z$ zkonstruujeme tah $T_i$ tak, že bude tým alokovat podle $p_Z(z_i)$.
  Stejně tak postupujme pro záchranná vozidla, zkonstruujme tah $U_i$ tak, že pro $\forall a_i \in A$ bude vozidlo alokovat podle $p_A(a_i)$.
  Nechť $T$ libovolná permutace posloupnosti tahů $T_i$ a $A_j$, $i \in \{ 1, \dots , Z_n\}, j \in \{ 1, \dots , A_n \}$.
  Pak $T(p_0) = p$.
\end{dukaz}

Vystačili jsme si pouze s tahy alokací týmu a vozidla.
Tah prodloužení směny jsme pro vyjádření libovolného plánu nepotřebovali.
To proto, že posloupnost tahů alokace týmu $z$ a následně prodloužení jeho směny je stejný tah, jako rovnou naalokování týmu $z$ s již prodlouženou směnou.
Závádění tahu prodloužení směny ale má svůj význam, více v (TODO).

\begin{veta}[Inverzní kanonické tahy]
  Nechť kanonické tahy alokování týmu $z$ tah $T_1$, alokování vozidla $a$ tah $T_2$ a prodloužení směny $z$ o $d$ tah $T_3$.
  Inverzní tahy k nim existují a jsou:

  \begin{enumerate}
    \item
      dealokování týmu $z$ tah $T_1^{-1}$,

    \item
      dealokování vozidla $a$ tah $T_2^{-1}$,

    \item
      zkrácení směny $z$ o $d$ tah $T_3^{-1}$.
  \end{enumerate}
\end{veta}
\begin{dukaz}
  První ukažme, že kanonické tahy splňují nutné podmínky pro existenci inverzního tahu \ref{veta:inverzNutnost}.

  Pro $T_1, T_2, T_3$ platí, že pokud se $T_i(p_1) = T_i(p_2)$, tak se muselo jednat o stejné plány už před provedením tahu a o různé, pokud $T_i(p_1) \neq T_i(p_2)$.
  Pro $T_1$ je dealokování týmu přesná konstrukce inverzního tahu z důkazu existence inverzní funkce \ref{veta:inverzNutnost}.
  Stejně tak pro $T_2$ a $T_3$.

  Inverzní tahy k $T_1, T_2$ a $T_3$ existují a jedná se právě o $T_1^{-1}, T_2^{-1}, T_3^{-1}$.
\end{dukaz}

Tah je prostředek pro převádění mezi plány.
Tahy umíme provádět opakovaně a zároveň $\forall p \in P_C$ existuje posloupnost kannonických tahů $T = T_1 \circ \dots \circ T_n$ taková, že $T(p_0) = p$ \ref{veta:vztahMeziPlanyAKan},
jinými slovy, existují tahy takové, že každý $p \in P_C$ je z $p_0$ nějakou jejich posloupností \textit{dosažitelný}.

\begin{definice}[Dosažitelnost tahy]
  Nechť tahy $T_i$.
  Plán $p_2$ je dosažitelný tahy $T_i$ z plánu $p_1$, pokud existuje posloupnost tahů $T$ taková, že $T(p_1) = p_2$.
\end{definice}

Pomocí tahů můžeme $P_C$ \textit{prohledávat}.
Pokud optimální plán $p^* \in P_C$ je nějakými tahy $T_i$ dosažitelný, pak ho můžeme prohledáváním nalézt.
Pomocí tahů můžeme $P_C$ prohledávat sofistikovaněji, než je jenom všechny procházet, jak jsme dělali v naivním řešení \ref{df:naivniRes}.
Některé plány totiž ani nemá smysl prohledávat, jako například plány, které naalokují záchranáře tak nešikovně, že neodbavují úspěšně žádný incident.
Taková naalokování jenom přispívají do ceny plánu, ale nijak do počtu úspěšně odbavených incidentů.
Jaké plány má smysl procházet?

\begin{definice}[Plán optimální v ceně]\label{veta:planOptVCene}
  Nechť $I$ množina incidentů a $p \in P_C$.
  Řekneme, že plán $p$ je optimální v ceně právě tehdy, když úspěšně odbaví všechny incidenty, $s_c(p, I) = |I|$ a zároveň
  pro plán $p'$ vzniklý z $p$ provedením libovolného inverzního kanonického tahu platí: $s_c(p', I) < |I|$.
\end{definice}

\begin{veta}[Optimální plán je optimální v ceně]
  Nechť libovolný optimální plán $p^* \in P_C$ při účelové funkci $q \in Q_I$ pro danou množinu incidentů $I$, pro který platí $s_c(p, I) = |I|$.
  Pak $p^*$ je optimální v ceně.
\end{veta}
\begin{dukaz}
  Pro spor předpokládejme, že $p^*$ není optimální v ceně, takže existuje $T^{-1}$ inverzní kanonický tah takový, že $p'^* = T^{-1}(p^*) \colon s_c(p'^*) = |I|$.
  Při dealokaci týmu, dealokaci vozidla i při zkracování směny snižujeme cenu plánu. To plyne přímo z definice ceny plány \ref{df:cenaPlanu}.
  Tudiž libovolný inverzní kanonický tah vždy sníží cenu plánu. 
  Plán $p'^*$ by tak byl plán s nižší cenou než $p^*$, ale zároveň by odbavoval stejný počet incidentů.
  Vzhledem k návrhu $q \in Q_I$ \ref{kap:opt1Uc}, by pak $p'^*$ musel být optimální, což je spor.
\end{dukaz}

\begin{definice}[Optimální tahy]
  Nechť množiny incidentů $I_1, \dots, I_n$ podle definice \ref{df:INC}.
  Nechť množiny $P_{k}$ všechny optimální plány v ceně na incidentech $I_k$, $\forall k \in \{ 1, \dots, n \}$.
  Definujme optimální tahy $T^*_{k-1}$ jako takové tahy, že $ P_k = \{ T^*(p) \mid \forall p \in P_{k-1}, \forall T^* \in T^*_{k-1} \}$, $\forall k \in \{ 2, \dots, n \}$.
\end{definice}


%  \begin{table}[h]
%    \begin{tabularx}{0.4\textwidth} { 
%      | >{\centering\arraybackslash}c 
%      | >{\centering\arraybackslash}X 
%      | >{\centering\arraybackslash}X | }
%    \hline
%    & $p_Z$ & $p_D$ \\
%    \hline
%      $z_1$  & $v_1$  & $(d_1, d_2)$  \\
%      $z_2$  & $v_0$  & $(0, 0)$  \\
%    \hline
%    \end{tabularx}
%    \quad
%    \begin{tabularx}{0.2\textwidth} { 
%      | >{\centering\arraybackslash}c 
%      | >{\centering\arraybackslash}X | }
%    \hline
%    & $p_A$ \\
%    \hline
%    $a_1$  & $v_1$ \\
%    $a_2$  & $v_0$ \\
%    \hline
%    \end{tabularx}
%
%    \caption{Optimální plán $p_{2}^*$}\label{tab:optPodplanNeniOptPlan}
%
%  \end{table}



%  Kanonický tah alokace týmu $z'$ a záchranného vozidla $a'$ na výjezdovou stanici je funkce $M_{z'a'} : p \rightarrow p'$, kde
%  \begin{align*}
%    p_Z'(z) &= p_Z^T(z) \in T_{z'}(p), \forall z \in Z,
%    \\
%    p_D'(z) &= p_D^T(z) \in T_{z'}(p), \forall z \in Z,
%    \\
%    p_A'(a) &=
%    \begin{cases}
%      p_Z'(z) & \text{pro záchranné vozidlo $a' \in A \colon p_A(a') = v_{\emptyset}, v \in V$.}, \\
%      p_A(a) & \text{pro $\forall a \in A \setminus \{ a' \}$},
%    \end{cases}
%  \end{align*}
%
%  pro $p_Z, p_D, p_A \in p$ a $p_Z', p_D', p_A' \in p'$, kde $p, p' \in P_C$ a $M_{z'a'}(p) = p'$.


% \section{Heuristické přístupy}\label{kap:heur}
