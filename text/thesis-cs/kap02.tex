\chapter{Řešení optimalizační úlohy}

Z minulé kapitoly již víme, že nelze vyzkoušet všechny plány a nalézt tak nejkvalitnější, protože velikost $P_C$ je příliš velká.
Víceúčelová funkce $q$ není lineární a není ani derivovatelná, jelikož interně spouští $S_I$, simulaci plánu na množině $I$.
Nesplňuje tak předpoklady pro použití metod prvního ani druhého řádu.
Optimalizační úlohu nelze vyjádřit ve tvaru soustav lineárních rovností a nerovností, takže ani jakékoliv techniky ze světa linéárního programování nám nejsou k dispozici.
Jediné co máme k dispozici jsou hodnoty účelove funkce $q$ v bodech $p$.
Pokud nebudeme schopni nalézt rekurzivní vztah, který budeme schopni pomocí memoizace vyřešit v pseudo polynomiálním čase pomocí dynamického programování, tak nejsem schopni
s jistotou nalézt optimální řešení a budeme se muset uchýlit k heuristickým přístupům.
Řešení přes dynamické programování podrobně popisuji (\textbf{odkaz}).
V práci se především věnuji metaheuristickým prohledávacím metodám, %TODO pak tady více rozved
jak metodám prohledávajících prostor lokálně, jako tabu prohledávání až po populační metody, jako genetické algoritmy.

\section{Prostor plánů pohotovostních služeb}

\section{Dynamické programování}

\section{Náhodná procházka}

\section{Hladové hledání nejbližšího lokálního minima}

\section{Tabu hledání}

\section{Simulované žíhání}

\section{Genetické algoritmy}


% další metody sem
