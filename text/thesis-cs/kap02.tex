\chapter{Řešení optimalizační úlohy}\label{chap:reseniOptUloh}

Víceúčelová funkce $q$ není derivovatelná.
Nesplňuje tak předpoklady pro použití metod prvního ani druhého řádu.
Optimalizační úlohu nelze vyjádřit ve tvaru soustav lineárních rovností a nerovností, takže ani jakékoliv techniky ze světa linéárního nebo nelineárního programování nám nejsou k dispozici.
Jediné co máme k dispozici jsou hodnoty účelove funkce $q$ v bodech $p$.
\\
\\
V této kapitole vyzkoušíme různé metody, které hledají globální maximum funkce, které můžeme pro nalezení optimálního plánu pohotovostní služby použít.

\section{Dynamické programování}\label{kap:dynamicProgram}

\section{Metaheuristické prohledávání}\label{kap:metaheur}

\subsection{Náhodná procházka}

\subsection{Hladové hledání nejbližšího lokálního minima}

\subsection{Tabu hledání}

\subsection{Simulované žíhání}

\subsection{Genetické algoritmy}

