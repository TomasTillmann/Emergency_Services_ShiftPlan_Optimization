\chapter{Řešení optimalizační úlohy}\label{chap:reseniOptUloh}

V této kapitole vyzkoušíme různé metody, které můžeme pro nalezení optimálního plánu pohotovostní služby použít.

\section{Dynamické programování}\label{kap:dynamicProgram}

\textit{Dynamické programování} je technika řešení problému, která si průběžně ukládá řešení menších podúloh a pomocí rekurzivního vztahu menší podúlohy s větší,
definovaného \textit{Bellmanovou rovností}, řeší větší podúlohy efektivněji, až po vyřešení původní úlohy. 
Průběžnému ukládání výsledků podúloh se říká \textit{memoizace}.
Díky této memoizaci dynamické programování neprohledává prostor řešení duplicitně, a tak je často velmi efektivní metodou pro řešení optimalizačních úloh (\citet{dynamic}).
Triviálním příkladem využití dynamického programování je výpočet $n$-tého \textit{Fibbonaciho čísla}. (\citet{mares}, kap. 12).

Pro nás zajímavějším příkladem využití dynamického programování je řešení úlohy kombinatorické optimalizace \textit{Problému batohu}.

\begin{definice}[Problém batohu]
  Nechť $n$ počet předmětů, které chceme vložit do batohu s kapacitou $c$.
  Každý předmět $i$ má výdělek $p_i$ a váhu $w_i$. Problém batohu pak je,

  \begin{alignat*}{2}
    &\normalfont \text{maximalizuj} \hspace{30pt} &z = \sum_{i=1}^n p_i x_i \\
    \\
    &\normalfont \text{splňující} \hspace{30pt} &\sum_{i = 1}^{n} w_i x_i \leq c,
  \end{alignat*}
  \\
  kde $x_i = 1$, pokud předmět $i$ je v batohu, jinak $0$.
  \\
\end{definice}

Existuje $2^n$ možností, které předměty vložíme do batohu. Naivní řešení prohledání všech možností tak běží v čase $\mathcal{O} (2^n)$.
Avšak pomocí dynamického programování lze vyřešit problém batohu v \textit{pseudopolynomiálním} čase $\mathcal{O}(nc)$,
což pro velká $n$ a konstatní $c$ je až exponenciálním zlepšením.

Bellmanovy rovnice vyjadřující rekurzivní vztahy vypadají následovně,

\begin{definice}[Rekurzivní vztah pro problém batohu]\label{rov:KPrekurz}
  \begin{align*}
    m_{i, c'} &= m_{i - 1, c'} \text{ pokud } w_i > c', \\
    m_{i, c'} &= \max (m_{i - 1, c'}, m_{i - 1, c' - w_i} + p_i) \text{ pokud } w_i \leq c',
  \end{align*}
  \\
  pro $0 \leq c' \leq c$ aktuální uvažovaná kapacita batohu.
\end{definice}

Rekurzivní vztah \ref{rov:KPrekurz} nám pouze říká, že ze znalostí optimálního výběru předmětů podpbroblému $m_{i-1, c' - w_i}$ a podproblému $m_{i - 1, c'}$,
umíme v konstantím čase zjistit optimální výběr předmětů pro aktuální problém $c'$, tedy $m_{i, c'}$.
Buď předmět $i$ použijeme při výběru, a tak aktuální optimální výběr je roven $m_{i - 1, c' - w_i} + p_i$,
nebo předmět při výběru nepoužijeme, takže aktuální optimální výběr je optimální výběr podproblému, $m_{i - 1, c'}$. 

\begin{algorithm}[h]
  \begin{algorithmic}[1]
  \Function{KnapsackProblem}{$n$, $c$, $p_i$, $w_i$, $1 \leq i \leq n$}
    \State $m_{i, j}$ \gets 0, $0 \leq i \leq n$, $0 \leq j \leq c$
    \For{$1 \leq i \leq n$}
      \For{$1 \leq j \leq c$}
        \If{$w_i > j$}
          \State $m_{i, j}$ \gets $m_{i - 1, j}$
        \Else
          \State $m_{i, j}$ \gets $\max$ ($m_{i - 1, j}$, $m_{i - 1, j - w_i} + p_i$)
        \EndIf
      \EndFor
    \EndFor
    \State \Return $m_{n,c}$
  \EndFunction
  \end{algorithmic}
  \caption{Problém batohu}
  \label{KP}
\end{algorithm}

Algoritmus \ref{KP} vrací sumu hodnot optimálních předmětů přidaných do batohu.
Jaké konkrétní předměty přispěly do sumy lze snado zjistit zpětným následováním rekurzivního vztahu \ref{rov:KPrekurz}.

Rádi bychom našli nějaký podobný rekurzivní vztah v problému hledání optimálního plánu.
Nalezením rekurzivních vztahů by pak simulace $s$ nebyla černou skříňkou a byla by splněna nutná podmínka \ref{veta:simulaceJakoCernaSkrinka} pro šanci na polynomiální řešení.

Nechť množina incidentů $I = \{ i_1, \dots , i_n \}$, kde $\forall j, k \in \{ 1, \dots n\} \colon T_I(i_j) \leq T_I(i_k)$, čili incidenty jsou seřazené podle času nastání.
Definujme množiny incidentů $I_1, I_2, \dots , I_n$, kde $\forall k \in \{ 1, \dots, n \} \colon I_k = \{ i_1, \dots, i_k \}$.
Incidenty $I_k$ jsou tak incidenty $I_{k-1}$ po odebrání incidentu $i_k$, který se odehrál jako poslední.
Nechť $p_{k-1} \in P_C$ libovolný optimální plán na množině incidentů $I_{k-1}$.
Nadále budeme zkoumat otázku, zda jsme schopni ze znalosti $p_{k-1}$ nalézt optimální plán $p_k$ na množině incidentů $I_k$.
V první řadě si potřebujeme nadefinovat jak obecně z nějakého plánu $p_1$ získat plán $p_2$. 

\begin{definice}[Tah]\label{df:tah}
  Nadefinujme tah jako funkci $T \colon P_C \rightarrow P_C$.
  Nadefinujme tahy
  \begin{enumerate}
    \item
      Alokace týmu na výjezdovou stanici.

    \item
      Alokace týmu a alokace záchranného vozidla na stejnou výjezdovou stanici.

    \item
      Prodloužení doby trvání směny již naalokovaného týmu záchranářu. 
  \end{enumerate}

  Formálně, tah alokace týmu $z'$ na výjezdovou stanici je funkce $M_{z'} : p \rightarrow p'$, kde
  \begin{align*}
    p_Z'(z) &=
      \begin{cases}
        v \neq v_{\emptyset} & \text{pro tým $z' \in Z \colon p_Z(z') = v_{\emptyset}, v \in V$}, \\
        p_Z(z) & \text{pro $\forall z \in Z \setminus \{ z' \}$},
      \end{cases}
      \\
    p_D'(z) &=
      \begin{cases}
        (d_1, d_2) \in D \colon d_2 - d_1 > 0 & \text{pro $z'$}, \\
        p_D(z) & \text{pro $\forall z \in Z \setminus \{ z' \}$},
      \end{cases}
      \\
    p_A'(a) &= p_A(a), \forall a \in A,
  \end{align*}

  pro $p_Z, p_D, p_A \in p$ a $p_Z', p_D', p_A' \in p'$, kde $p, p' \in P_C$ a $M_{z'}(p) = p'$.

  Tah alokace týmu $z'$ a záchranného vozidla $a'$ na výjezdovou stanici je funkce $M_{z'a'} : p \rightarrow p'$, kde
  \begin{align*}
    p_Z'(z) &= p_Z^T(z) \in T_{z'}(p), \forall z \in Z,
    \\
    p_D'(z) &= p_D^T(z) \in T_{z'}(p), \forall z \in Z,
    \\
    p_A'(a) &=
    \begin{cases}
      p_Z'(z) & \text{pro záchranné vozidlo $a' \in A \colon p_A(a') = v_{\emptyset}, v \in V$.}, \\
      p_A(a) & \text{pro $\forall a \in A \setminus \{ a' \}$},
    \end{cases}
  \end{align*}

  pro $p_Z, p_D, p_A \in p$ a $p_Z', p_D', p_A' \in p'$, kde $p, p' \in P_C$ a $M_{z'a'}(p) = p'$.

  Tah prodloužení směny týmu $z'$ je funkce $M_{z'd} : p \rightarrow p'$, kde
  \begin{align*}
    p_Z'(z) &= p_Z(z), \forall z \in Z \\
    p_D'(z) &=
      \begin{cases}
        (d'_1, d'_2) \in D \colon d'_2 > d_2 \land d'_1 = d_1  & \text{pro $z'$, kde $(d_1, d_2) = p_{D}(z)$} \\
        p_D(z) & \text{pro $\forall z \in Z \setminus \{ z' \}$},
      \end{cases}
      \\
    p_A'(a) &= p_A(a), \forall a \in A,
  \end{align*}

  pro $p_Z, p_D, p_A \in p$ a $p_Z', p_D', p_A' \in p'$, kde $p, p' \in P_C$ a $M_{z'd}(p) = p'$.

\end{definice}

%  \begin{table}[h]
%    \begin{tabularx}{0.4\textwidth} { 
%      | >{\centering\arraybackslash}c 
%      | >{\centering\arraybackslash}X 
%      | >{\centering\arraybackslash}X | }
%    \hline
%    & $p_Z$ & $p_D$ \\
%    \hline
%      $z_1$  & $v_1$  & $(d_1, d_2)$  \\
%      $z_2$  & $v_0$  & $(0, 0)$  \\
%    \hline
%    \end{tabularx}
%    \quad
%    \begin{tabularx}{0.2\textwidth} { 
%      | >{\centering\arraybackslash}c 
%      | >{\centering\arraybackslash}X | }
%    \hline
%    & $p_A$ \\
%    \hline
%    $a_1$  & $v_1$ \\
%    $a_2$  & $v_0$ \\
%    \hline
%    \end{tabularx}
%
%    \caption{Optimální plán $p_{2}^*$}\label{tab:optPodplanNeniOptPlan}
%
%  \end{table}









% \section{Heuristické přístupy}\label{kap:heur}
