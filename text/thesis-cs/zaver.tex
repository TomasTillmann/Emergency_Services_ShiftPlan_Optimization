\chapter*{Závěr}
\addcontentsline{toc}{chapter}{Závěr}

V této práci jsme zformalizovali problém nalezení optimálního plánu záchranné služby a následně jsme navrhnuli několik metod řešící tuto úlohu.
Nalezené metody jsme aplikovali na model Pražské záchranné služby a náhodně vygenerované sadě incidentů odpovídající, jak by se incidenty v Praze mohli reálně odehrávat.
Při analýze metody prohledávání plánů optimálními tahy jsme zjistili, že metoda je schopna poměrně rychle (7 minut) nalézt velmi kvalitní plán.
Lokální prohledávání z prázdného plánu je mnohem horší, a neumí z prázdného plánu nalézt podobně dobrý plán ani do hodiny.
Přednost lokálního prohledávání je v schopnosti doladit nějaký dost dobrý plán.
Kombinací lokálního prohledávání a prohledávání optimálními tahy jsme získali metodu,
která umí do 20 minut nalézt plán odbavující všechny incidenty a do hodiny dokonce lokálně optimální plán.

Tabu prohledávání není vhodnou metodou při řešení našeho problému, protože je obtížné najít i jenom lokální optimum.
Simulované žíhání narozdíl od tabu prohledávání je vhodná metoda a podobně jako u lokálního prohledávání nejlépe funguje v kombinaci s metodou prohledávání optimálními tahy. 

Jakou metodu použít velmi záleží na konkrétním cíli.
Pokud je žádoucí nalézt skutečně co nejlepší plán, odbavující nějakou danou sadu incidentů, tak je nejlepší zvolit kombinaci prohledávání optimálními tahy spolu s lokálním prohledáváním.
Ta nalezne velmi kvalitní plán, ale trvá poměrně dlouho.
Pokud je žádoucí nalézt dostatečně dobrý plán za kratší dobu, je lepší volbou spustit metodu prohledávání optimálními tahy spolu se simulovaným žíháním.
Simulované žíhání je rychlejší a bylo schopné nalézt kvalitní plán z plánu optimálního v ceně do 20 minut.

Nevýhodou prohledávání optimálními tahy je, že umí používat jenom účelovou funkci $q^{\text{Lex}}$.
Pokud bychom chtěli používat jinou účelovou funkci, například proto, že chceme nalézt plány, které odbavují jen o něco méně incidentů, ale jsou o dost levnější,
nemůžeme prohledávání optimálními tahy použít a musíme se spokojit pouze s metaheuristickými metodami.
V takovém případě je nejlepší z prázdného plánu nalézt dost dobrý plán simulovaným žíháním a následně nalézt jeho lokální optimum pomocí lokálního prohledávání.

Pražská záchranná služba už je poměrně větší záchranná služba a spolu s použitím Google API a větší sady incidentů se jednalo o výpočetně náročnou úlohu, avšak i tak
navrhnuté metody, pokud jsou správně použity, umí nalézt velmi kvalitní plány nejhůře do hodiny.

